\documentclass[12pt]{article}

\usepackage{sbc-template}

\usepackage{latexsym}

\usepackage[utf8]{inputenc}
\usepackage[english]{babel}

\usepackage{amsfonts,amsmath,amssymb,amsbsy,enumerate}
%\usepackage{graphics,graphicx,subfigure,psfrag}

\usepackage[mathscr]{euscript}
\usepackage{pstricks}
\usepackage{multirow}
\usepackage{verbatim}
\usepackage{color}
\usepackage{graphics,graphicx,subfigure,psfrag}
\usepackage{tabularx, multirow, booktabs}

\newtheorem{proposition}{Proposition}
\newtheorem{theorem}{Theorem}


\newcommand{\formulation}{\mathscr{F}}
\newcommand{\relaxation}{\mathscr{R}}
\newcommand{\V}{\chi}
\newcommand{\poli}{\mathcal{P}}
\newcommand{\poliRelax}{\mathcal{Q}}
\newcommand{\conv}{{\rm conv}}
\newcommand{\opt}{{\rm OPT}}
\newcommand{\nvar}{\eta}
\newcommand{\nullvec}{\pmb{0}}
\newcommand{\unitvec}{\pmb{1}}
\newcommand{\dist}{{\rm dist}}

\newcommand{\vertical}{|}
\newcommand{\aset}[1]{\left\{#1\right\}}

\newcommand{\DTIME}{\text{\textsc{DTIME}}}
\newcommand{\MinPC}{\text{\textsc{MinPC}}}
\newcommand{\MaxPC}{\text{\textsc{MaxPC}}}
\newcommand{\MinTSP}{\text{\textsc{MinTSP}}}
\newcommand{\MinNtPC}{\text{\textsc{MinNtPC}}}
\newcommand{\MaxNtPC}{\text{\textsc{MaxNtPC}}}
%\newcommand{\dCDS}{d\text{-\textsc{CDS}}}	
%\newcommand{\CVC}{\textsc{CVC}}
%\newcommand{\NP}{\mathscr{NP}}
\newcommand{\NP}{\text{\textsc{NP}}} %%%% new 
\newcommand{\Pclass}{\textsc{P}}  %%% new
%\newcommand{\Pclass}{\mathscr{Pclass}}
\newcommand{\APX}{\mathscr{APX}}
%\newcommand{\NP}{\textscr{NP}}
\newcommand{\FPT}{\textsc{fpt}}
\newcommand{\AP}{\textsc{ap}}
\newcommand{\Sep}{\mathcal{S}}

\newcommand{\R}{\mathbb{R}}
\newcommand{\Rn}{\mathbb{R}^n}
\newcommand{\Z}{\mathbb{Z}}
\newcommand{\B}{\mathbb{B}}
\newcommand{\N}{\mathbb{N}}
\newcommand{\Q}{\mathbb{Q}}

\newcommand{\bigO}{\mathcal{O}}
\newcommand{\littleO}{o}
\newcommand{\NonCutVertices}{\overline{\mathcal{C}}}

\newcommand{\otoprule}{\midrule[\heavyrulewidth]}	


%----hugo
\usepackage{mathrsfs} %mathscr
\newcommand{\smallG}{\mbox{\larger[-1]$(G)$}}
\newcommand{\smallGf}{\mbox{\larger[-1]$(G-f)$}}
\newcommand{\smallGprime}{\mbox{\larger[-1]$(G')$}}
\newcommand{\smallGprimef}{\mbox{\larger[-1]$(G'-f)$}}
\newcommand{\smallGprimeG}{\mbox{\larger[-1]$(G',G)$}}
\newcommand{\smallW}{\mbox{\larger[-1]$(G[W])$}}
\newcommand{\smallWg}{\mbox{\larger[-1]$(G[W]-g)$}}
\newcommand{\smallWWbarf}{\mbox{\larger[-1]$(G[E(W) \cup E(\overline{W})+f])$}}
\newcommand{\smallWWbarfg}{\mbox{\larger[-1]$(G[E(W) \cup E(\overline{W})+f-g])$}}
\newcommand{\smallH}{\mbox{\larger[-1]$(H)$}}
\newcommand{\smallF}{\mbox{\larger[-1]$(G[F])$}}
\newcommand{\spanPath}{\mathcal{P}}
\newcommand{\spanBridge}{\mathcal{B}}
\newcommand{\Pathuv}{\spanPath_{u,v}^{t}}
\newcommand{\Pathpq}{\spanPath_{p,q}^{t}}
\newcommand{\Path}{\spanPath^{t}}
\newcommand{\PathuvF}{\Pathuv\smallF}
\newcommand{\PathuvG}{\Pathuv\smallG}
\newcommand{\PathuvGprime}{\Pathuv\smallGprime}
\newcommand{\PathuvGf}{\Pathuv\smallGf}
\newcommand{\PathuvGprimef}{\Pathuv\smallGprimef}
\newcommand{\PathpqW}{\Pathpq\smallW}
\newcommand{\PathpqWg}{\Pathpq\smallWg}
\newcommand{\PathpqWWbarf}{\Pathpq\smallWWbarf}
\newcommand{\PathpqWWbarfg}{\Pathpq\smallWWbarfg}
\newcommand{\BridgeuvPrime}{\spanBridge_{u',v'}^{t}}
\newcommand{\BridgeuvPrimeG}{\spanBridge_{u',v'}^{t}(G)}
\newcommand{\Bridgeuv}{\spanBridge_{u,v}^{t}}
\newcommand{\BridgeuvG}{\spanBridge_{u,v}^{t}(G)}
\newcommand{\Bridge}{\spanBridge^{t}}
\newcommand{\BridgeG}{\spanBridge^{t}(G)}
\newcommand{\smallQBridge}{\mbox{\larger[-1]$(Q \cup \Bridge)$}}
\newcommand{\PathuvQBridge}{\Pathuv\smallQBridge}
\newcommand{\BridgeuvGprime}{\Bridgeuv\smallGprime}
\newcommand{\BridgeuvPrimeGf}{\spanBridge_{u',v'}^{t}\smallGf}
\newcommand{\BridgeuvGf}{\Bridgeuv\smallGf}
\newcommand{\BridgeGf}{\Bridge\smallGf}
\newcommand{\BridgeuvGprimeG}{\Bridgeuv\smallGprimeG}
\newcommand{\BridgeTwo}{\spanBridge_{2}^{t}}
\newcommand{\BridgeGprime}{\Bridge\smallGprime}
\newcommand{\BridgeTwoGprime}{\BridgeTwo\smallGprime}
\newcommand{\BridgeTwoGf}{\BridgeTwo\smallGf}


%LP formulation
\usepackage{lpform} %LPs
\renewcommand{\lpforall}[1]{&& \forall #1}
%\renewcommand{\lpsubjectto}{}
\renewcommand{\lpindent}{\hspace{0pt}}

%\newcommand{\dist}{{\rm dist}}
\newcommand{\maxcam}{{\rm maxCam}}
\newcommand{\gr}{{\rm gr}}

\newcommand{\arbset}{\Gamma}


%comandos para novas letras/exppressões
\newcommand{\incid}{\mathcal{X}}
\newcommand{\incidY}{\mathcal{Y}}
\newcommand{\maxstretchfactor}{\mathcal{T}}
\newcommand{\espacoX}{\mathbb{R}^{|E|}}
\newcommand{\espacoY}{\mathbb{R}^{|E| \times |E|}}
\newcommand{\espacoZ}{\mathbb{R}^{|V| \times |E| \times |E|}}
\newcommand{\espacoU}{\mathbb{R}^{|V| \times |V|}}
\newcommand{\espacoUuni}{\mathbb{R}^{|V|}}
\newcommand{\espacoXBinary}{\mathbb{B}^{|E|}}
\newcommand{\espacoYBinary}{\mathbb{B}^{|E| \times |E|}}
\newcommand{\espacoZBinary}{\mathbb{B}^{|V| \times |E| \times |E|}}
\newcommand{\PGtrestricoes}{\rm{(PL)}}
\newcommand{\facetF}{\mathcal{F}}

%\usepackage[pdftex]{graphicx} %pacote que esta dando problema
\usepackage{pdfpages}
\usepackage{newclude} 

\DeclareMathOperator*{\argmin}{arg\,min}

\usepackage{setspace}

%enumerate package
\usepackage{enumitem}

%evitar quebrar bloco
\newenvironment{absolutelynopagebreak}
  {\par\nobreak\vfil\penalty0\vfilneg
   \vtop\bgroup}
  {\par\xdef\tpd{\the\prevdepth}\egroup
   \prevdepth=\tpd}

%---------------

\sloppy

\title{Minimum Weight Tree Spanner
  Problem\protect\footnote{Preliminary results of an ongoing research
    carried out by the author in his PhD program at IME-USP under the
    supervision of Professor Yoshiko Wakabayashi. This work also
    benefited from helpful discussions with Professor Manoel Campêlo
    (UFC).  Research supported by FAPESP fellowship, Project
    (Proc. 2013/22875-9) and MaCLinC project of NUMEC/USP.}}

\author{Hugo Braga}

\address{Instituto de Matemática e Estatística -- Universidade de São Paulo\\
  05508-090 -- São Paulo -- SP -- Brazil
  \email{hbraga@ime.usp.br}
}

\begin{document} 

\maketitle

\begin{abstract}
  Let $(G,w,t)$ denote a triplet consisting of a connected graph~$G=(V,E)$
  with a nonnegative weight function $w$ defined on $E$, and a real
  number $t>1$.
  A tree \hbox{$t$-spanner} of $G$ is a spanning tree~$H$ of $G$
  such that for each pair of vertices~$u$,$\,v$, the distance between
  $u$ and $v$ in $H$ is at most $t$ times the distance between $u$ and
  $v$ in $G$.  
  We address the \emph{Minimum Weight Tree Spanner Problem 
  (MWTS)}, defined as follows. Given a triplet $(G,w,t)$, find a
  tree $t$-spanner in $G$ that has the smallest possible
  weight. It is known that \emph{MWTS} is {\rm NP}-hard for every fixed
  $t\ge 4$.
  We propose two ILP formulations for \emph{MWTS}, based on
  arborescences, of polynomial size, and present some preliminary
  results on the computational experiments with these formulations.
  %% We also comment
  %% on the implementation of an exact algorithm for the bounded diameter
  %% spanning tree problem, which is used as a preprocessing routine for
  %% the cardinality version of the problem.
\end{abstract}
     
\begin{resumo}
  Seja $(G,w,t)$ uma tripla formada por um grafo conexo $G=(V,E)$ com
  uma função peso $w$ definida sobre $E$, e um número real $t>1$.  Uma
  árvore $t$-spanner de $G$ é uma árvore geradora~$H$ de~$G$ tal que
  para quaisquer pares de vértices~$u$,$\,v$, a distância entre $u$ e
  $v$ em $H$ é no máximo $t$ vezes a distância entre $u$ e $v$ em $G$.
  Estudamos o \emph{problema da árvore spanner de custo mínimo},
  denotado por \emph{MWTS} (acrônimo de \emph{Minimum Weight Tree
    Spanner}): dada uma tripla $(G,w,t)$, encontrar em $G$ uma
  árvore $t$-spanner que tenha o menor peso possível. Sabe-se que
  \emph{MWTS} é {\rm NP}-difícil para todo $t\ge 4$, fixo.  Propomos duas
  formulações lineares inteiras para o \emph{MWTS}, baseadas em
  arborescências, de tamanho polinomial, e apresentamos resultados
  preliminares sobre os experimentos computacionais realizados com
  essas formulações.
  %% Também comentamos a implementação de um
  %% algoritmo para o problema da árvore gerador de diâmetro limitado,
  %% que é usado na fase de preprocessamento para a versão cardinalidade
  %% do problema.

\end{resumo}

%%%%%%%%%% INTRO %%%%%%%%
\section{Introduction}
%%%%%%%%%%%%%%%%%%%%%%
In this work, $(G,w,t)$ denotes a triplet consisting of a connected
graph $G=(V,E)$ with nonnegative weight $w_e$ for each edge $e\in E$,
and a real number $t>1$.  For two distinct vertices~$u$, $v$ in $V$,
the \emph{distance between $u$ and $v$} in $G$, denoted by 
$\dist_{G}(u,v)$, is the length of a shortest path (w.r.t.~$w$) from
$u$ to $v$ in $G$.  

A \emph{$t$-spanner} of $G$ is a spanning subgraph~$H$ of $G$ such
that $\dist_{H}(u,v) \le t\cdot\dist_{G}(u,v)$ for all vertices $u$,
$v$ in $V$.  The integer $t$ is called the \emph{strech factor}. A
\hbox{\emph{tree $t$-spanner}} is a $t$-spanner that is a tree.  We study the
\emph{Minimum Weight Tree Spanner Problem} (MWTS), defined as
follows: given a a triplet $(G,w,t)$, find a tree $t$-spanner in $G$
of minimum weight.

The cardinality version of this problem is basically a feasibility (or
existence) problem. Even in this case, the problem is known to be
NP-complete~\cite{CaiC1995} for every fixed $t\geq 4$.  It is
known to be solvable in polynomial time for $t=2$
~\cite{Bondy1989,CaiC1995}. The computational complexity
status when $t=3$ is unknown.

The more general problem, in which the objective is to find a
$t$-spanner, has been largely investigated. It arises in distributed
computing scenarios~\cite{Awerbuch1985}, communication
networks~\cite{PelegU1988} and robotics.  The concept of
\emph{spanners} was introduced in 1987 (in a conference) by Peleg and
Ullman~\cite{PelegU1988}, in connection with construction of synchronizers.
%  who showed that spanners could be used to
% construct synchronizers that transform synchronous algorithms into
% asynchronous ones.
Heuristics as well as a column-generation approach
have been proposed for the graph spanner (but not for the tree spanner) problem.

%  scenarios~\cite{Awerbuch1985,PelegU1989},
% networks~\cite{PelegU1988,PelegR1999,OliveiraP2005,Braga2012} 
% robotics~\cite{ArikatiCCDSZ1996}.

% \begin{equation}
% %\label{eq:spanner}
% \dist_{H}(u,v) \le t \cdot \dist_{G}(u,v), \quad \forall\;u,v \in V, 
% \label{eq:def_spanner}
% \end{equation}  
%

% When $H$ is a tree, it is called a \emph{tree $t$-spanner}. Let 
% $w: E(G) \to \mathbb{R}^+$ be the weight function associated with $G$. Due to 
% \cite{CaiC1995}, we also 
% can say that $H$ is a tree $t$-spanner of $G$ if 
%
% \begin{equation}
% \dist_{H}(u,v) \le t \cdot w(u,v) \;\forall\,uv \in E
% \label{eq:def_spanner2}
% \end{equation} 
% %$\dist_{H}(u,v) \le t \cdot w(u,v) \;\forall\,uv \in E$.
%
% A central problem related to spanners is called \emph{tree $t$-spanner
%   problem}. Given a graph~$G$ and a real number $t>1$, the aim is to
% decide if $G$  has a tree $t$-spanner. For $t=2$, this problem is
% polinomial \cite{Bondy1989,Cai1992,CaiC1995}. For fixed $t \ge 4$,
% it is NP-Complete \cite{Cai1992,CaiC1995}. The problem is open for $t=3$.
%
% Given a graph~$G$, a real $t>1$ and a weight function
% $w: E(G) \to \mathbb{R}^+$, int the \emph{Minimum Weight $t$-Spanner
%   Tree Problem} (MWSTP) we to find a tree $t$-spanner $H$ of $G$
% with minimum cost, i.e., that minimizes $\sum_{e \in E(H)} w(e)$.
% Determining the existence of a tree $t$-spanner of~$G$, for any fixed
% $t>1$ in a weighted graph is NP-Complete due to Cai and
% Corneil~\cite{CaiC1995}. As a consequence, the decision version of
% MWSTP is NP-Complete for $t>1$.
%
%% In Section~2 we present two integer linear programming formulation for
%% the MWSTP problem. Then, in Section~3 we report on some preliminary
%% computational experiments with these formulations.
% Owing to space limitation, we do not give much detail about the formulation. 
%
In this work we focus on two ILP formulations for MWTS.  In these
formulations, we use the result proved by~\cite{CaiC1995} that
guarantees that the stretch factor condition can be simplified to
$ \dist_{H}(u,v) \le t \cdot w(u,v)$ for all $uv\in E$.
%
To our knowledge, no approximation algorithms or ILP formulations have
been proposed for MWTS. 


\section{Integer Linear Formulations for MWTS}
The polyhedron of the tree spanner problem associated with
$(G,w,t)$ is defined as
$ P(G,t) := \text{conv}\{\incid^{F} \in \espacoX\; |\; \text{$G[F]$ is
  a tree $t$-spanner\}}$, where $\incid^{F}$ denotes the incidence
vector of $F$. In what follows we present two ILP formulations. With
respect to the two polyhedra defined by the convex hull of the integer
points satisfing these formulations, $P(G,t)$ is a projection on the
space~$\espacoX$. 

\subsection{Formulation 1: finding distances between vertices}
Given a graph $G=(V,E)$, let $D=(V,A)$ be the digraph obtained
from $G$, where $A = \{(u,v),(v,u): uv \in E\}$. Let
$\mathbb{B}:=\{0,1\}$. Take a vertex $v \in V$ (root of a \hbox{$v$-rooted}
arborescence) and consider the variable
$z^{v} = (z_{ij}^{v})_{ij \in A}$, associated with $v$. This variable
tells which arcs are in the $v$-rooted arborescence. 
%
In the formulation we deal with aborescences in $D$, rooted at
different vertices of this digraph. They allow us to find paths
between vertices on the given arborescences in a easy way. Then, using
these arborescences, we construct a spanning tree (defined by the
variables $x$) satisfying the stretch factor condition.

The decision variable $x \in \espacoXBinary$ has the following
meaning: for each edge~$e$, \hbox{$x(e) = 1$} if and only if $e$ is part of
the solution.  For each $f \in E$, consider
$y^{f} = (y^{f}_{e})_{e \in E}$. The variable $y \in \espacoYBinary$
has the following meaning: for each edge $e$, and edge \hbox{$f=uv$,}
$y^{f}_{e} = 1$ if and only if $e$ is in the path between $u$ and $v$
in the solution tree.
%
%We have the following ILP formulation for MWSTP:
%
%\begin{absolutelynopagebreak}
  \begin{lpformulation} %[{\rm (1)}]
    \lpobj*{min}{\sum_{e\in E} w_ex_e}
    \lpeq[res_mwstp:num_aresta]{\sum_{e \in E}x_e = |V| - 1}{}
    \lpeq[]{\sum_{i \in \delta^{-}(j)}z^{r}_{ij} = 1}{r \in V,\,\forall j \in V \setminus \{r\}}
    \lpeq[]{\sum_{i \in \delta^{-}(r)}z^{r}_{ir} = 0}{r \in V}
    \lplabel{lp:tree_restriction}
    \lpeq[]{x_{e} = z^{r}_{ij} + z^{r}_{ji}}{r \in V,\, \forall e = \{i,j\} \in E}
    \lpeq[]{z^{u}_{ij} - z^{v}_{ij} \le y^{uv}_{e} \le z^{u}_{ij} + z^{v}_{ij}} {uv \in E,\, \forall e = \{i,j\} \in E}
    \lpeq[]{z^{u}_{ji} - z^{v}_{ji} \le y^{uv}_{e} \le z^{u}_{ji} + z^{v}_{ji}} {uv \in E,\, \forall e = \{i,j\} \in E}
    \lpeq[res_mwsp:single_in-arc]{\sum_{e \in E} w^{}_e\,y^{uv}_{e} \le t \cdot w(u,v)}{uv \in E}
    \lpeq[]{x \in \espacoXBinary, y \in \espacoYBinary, z^{v} \in \espacoYBinary}{v \in V}
\end{lpformulation}
%\end{absolutelynopagebreak}

  \subsection{Formulation 2: with variables representing distances between pairs of vertices}
 
  For this formulation, we consider the same setting as the previous
  formulation. Given $G = (V,E)$, we consider similarly the digraph
  $D = (V,A)$.  The variables $z \in \espacoZ$ and $x \in \espacoX$
  are also defined similarly. Additionally, now let $u \in \espacoU$
  be a variable, such that for $i,j \in V$, when $u^{j}_{i} = u^{i}_{j}$, then
  $u^{i}_{j}$ represents the distance between $i$ and $j$. In the formulation,
  $M_{ij}$ is a given upper bound on the distance between vertices $i$
  and $j$ in~$G$.
  % $u^{r}_{i} - u^{r}_{j}$.  
%
  \begin{lpformulation} %[{\rm (2)}]
    \lpobj*{min}{\sum_{e\in E} w_ex_e}
    \lpeq[res_mwstp:num_aresta]{\sum_{e \in E}x_e = |V| - 1}{}
    \lpeq[]{\sum_{i \in \delta^{-}(j)}z^{r}_{ij} = 1}{r \in V,\,\forall j \in V \setminus \{r\}}
    \lpeq[]{\sum_{i \in \delta^{-}(r)}z^{r}_{ir} = 0}{r \in V}
    \lplabel{lp:tree_restriction}
    \lpeq[]{x_{e} = z^{r}_{ij} + z^{r}_{ji}}{r \in V,\, \forall e = \{i,j\} \in E}
    \lpeq[res:mtz]{u^{r}_{i} - u^{r}_{j} + (M_{ij} + w_{ij}) z^{r}_{ij} + (M_{ij} -w_{ij})z^{r}_{ji} \le M_{ij}}{r \in V, \forall ij \in A, j \ne r}
    \lpeq[]{u^{r}_{i} + (M_{ir} -w_{ir})z_{ri} \le M_{ir}}{r \in V, \forall ri \in A}
    \lpeq[]{u^{r}_{r} = 0}{r \in V}
    \lpeq[res:spanner]{u^{j}_{i} = u^{i}_{j} \le t \cdot w_{ij}}{ij \in E}
    \lpeq[]{x \in \espacoXBinary, z^{r} \in \espacoYBinary, u^{r} \in \espacoUuni}{r \in V}
\end{lpformulation}

The idea behind constraint (\ref{res:mtz}) was used by~\cite{MillerTZ60} for the TSP.
%For the unit graph, the following inequation is valid
%$\forall r \in V, rv \in A, u^{r}_{v} + z^{r}_{rv} \ge 2$. 

\section{Computational experiments}

We carried out some computational experiments to compare the two
formulations and to have some idea of the strength of these
formulations.  We focused the unweighted case (cardinality version)
and the case in which the weights represent Euclidean distance. For
the cardinality version, we implemented a polynomial-time algorithm
(showed by Nadiradze, 2013) for the \emph{Bounded Diameter Minimum
  Spanning Tree} problem to use in the preprocessing phase. We also
added two other valid inequalities for this version.

Our implementation was coded in C++, using CPLEX as the ILP solver.
The code was run on a machine with 65 GB of RAM and 1.6 GHz (using
single core).
% Although the machine is
% multiprocessor, we configured to use only a single core. 
% We implemented the two formulations and performed some experiments.
%  We implemented some graph algorithms using the C++
% library \emph{Lemon}. 
%
We present only results obtained with Formulation~2, as it
consistently outperformed Formulation~1.  Table~\ref{tab:exec_time}
shows results for the cardinality version. We considered three
parameters, and for each combination, we generated 10 random
instances.  \emph{Density} means the percentage of edges in the graph
compared to the complete graph. \emph{Time} is the average time
in seconds (for the instances solved with ILP); TLE (Time Limit Exceeded)
indicates the number of instances not solved within 1~hour of CPU
time.

% \begin{comment}

% \begin{center}
% \noindent
% \footnotesize{
% \begin{tabular}{|c|c|c|c|c|c|r|}\hline
% {$t$} & {$|V|$} & {Density} & {\# Solved} & {Time (s)} & {\# Solved
%                                                          with ILP} & {Time (s)}
% \\ \hline\hline
% \multirow{9}{*}{3} & 10 & 20 & - & - & 10 & 0.0050  \\
% & 10 & 40 & - & - & 8  & 0.0175 \\
% & 10 & 60 & -  & - & 1  & 0.0300   \\     
% & 20 & 20 & - & - & 10 & 0.0650 \\
% & 20 & 40 & - & - & 8  & 20.3025\\
% & 30 & 20 & - & - & 10 & 0.4090  \\
% & 30 & 40 & - & - & 3  & 3080.7934\\
% & 40 & 20 & - & - & 10 & 152.0970\\
% & 50 & 20 & - & - & 10 & 829.6290\\
% \hline\hline
% \end{tabular}
% }
% \captionof{table}{Computational results for the cardinality version}\label{tab:exec_time}
% \end{center}

% \end{comment}

\smallskip

\begin{center}
\noindent
\footnotesize{
\begin{tabular}{|c|c|c|c|c|c|r|}\hline
{$t$} & {$|V|$} & {Density} & {\# Solved in prep. phase} & {TLE}  &{\# Solved
                                                           with ILP}  &{Time (s)}
\\ \hline\hline
%\multirow{12}{*}{3} & 10 & 20 & 0 & 0 & 10 & 0.01  \\
\multirow{11}{*}{3} & 10 & 40 & 5 & 0 & 5  & 0.02 \\
& 10 & 60 & 10 & 0 & 0 & -   \\     
& 20 & 20 & 0 & 0 & 10 & 0.07 \\
& 20 & 40 & 0 & 0 & 10  & 7.03\\
& 20 & 60 & 7 & 0 & 3  & 87.61\\
& 30 & 20 & 0 & 0 & 10 & 0.32  \\
& 30 & 40 & 0 & 0 & 10  & 1250.94\\
& 30 & 60 & 7 & 3 & 0  & - \\
& 40 & 20 & 0 & 0 & 10 & 83.30\\
& 40 & 40 & 0 & 10 & 0 & -\\
& 40 & 60 & 0 & 10 & 0 & -\\
%& 50 & 20 & 0 & 3 & 7 & 4.58\\
%& 50 & 40 & 0 & 10 & 0 & -\\
%& 50 & 60 & 1 & 9 & 0 & -\\
%& 60 & 40 & 0 & 10 & 0 & -\\
%& 60 & 60 & 0 & 10 & 0 & -\\
%& 50 & 40 & - & - & XXXX  & TIRAR?  Tem dados?   \\
\hline\hline
\end{tabular}
}
\captionof{table}{Computational results for the cardinality version}\label{tab:exec_time}
\end{center}


For $t=3$, instances with density $40$ and $60$ are the hardest ones.
For this case, almost all instances (of order up to $60$) with
density~$20$ could be solved quickly.  Similarly, for $t\in \{3,4\}$,
all instances (of order up to~$60$) with density~$80$ were solved
quickly (all of them with preprocessing). For $t=4$, all instances
with density~$40$ or $60$ were solved quickly (with positive answer).

% We also considered instances for the cardinality version with density
% $20$ and $80$. For $t=3$, almost all instances with density~$20$ could
% be solved quickly.  Similarly, for $t\in \{3,4\}$, all instances with
% density~$80$ were solved quickly (all of them with preprocessing). For
% $t=4$, all instances with density~$40$ or $60$ were solved quickly
% too.
%% instances with density~$40$ of order up to $30$ could be solved
%% within 1~hour of CPU time;
%% but as the order increases, we were not able to
%% solve within this time limit. For large $t$, instances with density~40
%% in general admit solutions and it is easy to solve them (with
%% preprocessing).

For the Euclidean case, we considered instances of order in the range
from $10$ to $120$ (subgraphs of graphs from public libraries).  For
$t = 3$, all instances were solved. However, few of these instances
admit a tree \hbox{$3$-spanner}, and these are all of low order. For
$t = 4$, the greater the density, the smaller the number of instances
solved. In this case, for instances with density~$80$, only instances
of order up to~$40$ were solved.  For the instances considered,
finding tree spanners for $t=3$ was faster than for $t=4$. We have to
perform more computational experiments, but it seems that the random
instances, cardinality version, are harder to be solved. 

% For $t = 3$, the
% MWTS is solved in less time when compared to $t = 4$.
% %% When we compare the time, to solve the problem for $t = 4$ spends more time
% %% than  for $t = 3$.
% The explanation for both cases 
% %the results of these two comparisons
% is that the solver
% has more possibilities in the decision tree when $t = 4$.

\bibliographystyle{sbc}
\bibliography{Spanner}

\end{document}

