%!TEX root = main.tex
%% ------------------------------------------------------------------------- %%
\chapter{Introdução}
\label{cap:introducao}

Dado um grafo $G = (V,E)$, e um número real $t \ge 1$, um
\emph{$t$-spanner} de $G$ é um subgrafo gerador $H$ de $G$ tal que
para quaisquer pares de vértices $u$,$v$, a distância entre $u$ e $v$
em $H$ é no máximo $t$ vezes a distância entre $u$ e $v$ em $G$.  Este
conceito de $t$-\emph{spanner} pode ser generalizado para o caso em
que às arestas do grafo $G$ estão associados pesos não negativos, ou
seja, é dada uma função peso $w: E \to \espacoRpos$, e o conceito de
distância é o que comumente entendemos para grafos com pesos (ou
comprimentos) nas arestas. Quando o subgrafo $H$ é uma árvore, dizemos
que $H$ é uma \emph{árvore $t$-spanner}. Não encontramos uma tradução
que consideramos apropriada para o termo \emph{spanner}, e decidimos
manter o termo original, sem grafá-lo em itálico. 


Problemas sobre spanners surgem em vários cenários, em diferentes áreas, como computação distribuída~\cite{Awerbuch1985,PelegU1989}, redes de comunicação~\cite{PelegU1988,PelegR1999,OliveiraP2005} e robótica~\cite{ArikatiCCDSZ1996,MarbleB2011,MarbleB2011b}. No Capítulo~\ref{cap:conceitos_historico-v2}, definimos formalmente os principais problemas sobre  spanners e apresentamos os resultados sobre algoritmos conhecidos e complexidade computacional desses problemas. Além disso, apresentamos um histórico do estudo de spanner, desde o surgimento do conceito até os dias atuais. Antes de apresentar esses problemas, no Capítulo~\ref{cap:preliminares-v2}, apresentamos conceitos
  básicos de teoria dos grafos, combinatória poliédrica e otimização
combinatória para estabelecer a terminologia.


 % Essa sua aplicabilidade em diversas áreas tem motivado ampla pesquisa, envolvendo áreas como teoria dos grafos, otimização combinatória, e mesmo geometria computacional (nas variantes em que as distâncias são euclidianas). Este conceito foi introduzido em 1987, e desde então muita pesquisa tem sido desenvolvida na área de grafos e otimização combinatória. O tópico continua a ser de bastante interesse dos pesquisadores, como podemos ver em trabalhos na área de sistemas distribuídos~\cite{ElkinN2017,GrossmanP2017}, trabalhos que focam na esparsidade do spanner produzido~\cite{BodwinW2015,ElkinN2017,BorradaileLW2017}, e também em oráculos de distância~\cite{ElkinP2016,Sommer2016,Knudsen2017}.

Nesta tese focamos dois problemas sobre spanners: \emph{problema da árvore $t$-spanner de  peso mínimo} (cuja sigla em inglês é MWTSP), e \emph{problema de $t$-spanner de peso mínimo} (cuja sigla em inglês é MWSP). Em ambos os problemas, a entrada é uma tripla $(G,w,t)$.
No primeiro problema o objetivo é encontrar uma árvore $t$-spanner em $G$ de peso mínimo; e no segundo, o objetivo é encontrar um $t$-spanner em $G$ de peso mínimo. 

Ambos os problemas são NP-difíceis. O problema MWTSP já é NP-difícil mesmo para $t$ fixo, quando $t > 1$~\cite{CaiC1995}. O segundo problema, MWSP, já é NP-difícil mesmo quando os grafos têm peso unitário, e $t$ é fixo, $t \ge 2$~\cite{Cai1994}.

O problema da árvore $t$-spanner de peso mínimo (MWTSP) tem sido largamente investigado no caso em que os pesos são unitários. Para este caso há na literatura alguns algoritmos e resultados sobre a complexidade computacional para classes específicas de grafos. Identificamos uma lacuna com relação à existência de um algoritmo exato para o caso de pesos arbitrários, e em vista disso, optamos por desenvolver um tal algoritmo. 

% os autores estavam interessados em resultados de complexidade para classes gerais e específicas de grafos.
% Em decorrência de soluções exatas específicas para o MWTSP, optamos por desenvolver algoritmos exatos para este problema.

No Capítulo~\ref{cap:mwstp}, propomos duas formulações lineares inteiras para o MWTSP.
Estas são as primeiras formulações lineares inteiras específicas
  para o problema da árvore $t$-spanner. Diferentemente de formulações
  propostas para o problema de grafos $t$-spanner, as formulações que 
  propusemos são  específicas para  árvores $t$-spanner (no sentido de não serem formulações para grafos spanners com a exigência adicional de aciclicidade).  Além disso, as formulações lineares inteiras propostas por nós
  para o MWTSP  são polinomiais no tamanho de $G$, diferentemente das formulações existentes
para o MWSP.
Ambas as formulações são baseadas em arborescências que ``se sobrepõem''. Esta ideia facilita encontrar o caminho entre dois vértices quaisquer nas arborescências (e, por consequência, na árvore que representa uma solução). As duas formulações diferem quanto à forma como a distância entre os pares de vértices é armazenada na formulação.  Em uma das formulações, esta distância é calculada de maneira explícita.  Numa segunda formulação, existem variáveis que representam a distância entre pares de vértices. Esta segunda formulação mostrou-se melhor nos experimentos computacionais realizados. 

Nas formulações lineares inteiras existentes para o MWSP, geralmente os autores definem variáveis para cada caminho $t$-spanner~\cite{SigurdZ2004,DinitzK2011}. Neste caso, se todas as colunas forem consideradas desde o início, a formulação terá um número exponencial (no tamanho de $E$) de colunas. Neste caso, alguns autores como Sigurd e Zachariasen~\cite{SigurdZ2004} adotam o método de geração de colunas. Na formulação linear inteira que propomos no Capítulo~\ref{cap:mwsp} para o MWSP, adotamos variáveis associadas às arestas e aos pares de arestas. Nesta formulação, o número de variáveis é polinomial no tamanho de $G$, mas o número de restrições é exponencial no tamanho de $E$.  Para tratar as restrições de tamanho exponencial, do programa linear relaxado correspondente, precisamos resolver o problema da separação. No problema da separação, dado um conjunto de inequações e uma (possível) solução, queremos saber se existe uma inequação violada pela solução.  Em outras palavras, desejamos saber se a solução é viável ou obter um certificado que corresponda a uma inequação violada.  A separação das restrições cujo tamanho é exponencial (do programa linear relaxado) pode ser feita em tempo polinomial no tamanho de $V$, visto que nós mostramos que o problema da separação corresponde ao problema de corte mínimo, que pode ser resolvido por um algoritmo de tempo polinomial.
Apesar desta proposta diferente (quando comparada às formulações
  lineares inteiras existentes), os experimentos iniciais mostraram que os
  resultados não foram bons (e não serão apresentados). O fortalecimento da
  formulação proposta por nós, através da descoberta
de inequações válidas, é uma alternativa para melhorar os resultados.

Sigurd e Zachariasen~\cite{SigurdZ2004} propuseram um algoritmo exato para o MWSP, com o intuito de verificar quão boa era a heurística gulosa proposta por Alth\"{o}fer et al~\cite{AlthoferDDJS1993}. Os autores propõem uma formulação linear inteira cujo número de variáveis é exponencial (no tamanho de $E$). Em decorrência disso, os autores adotam um método de \emph{branch-and-bound} (B\&B) aliado à geração de colunas para a resolução do problema. O problema do \emph{pricing} relacionado à geração de colunas é o \emph{problema do caminho mínimo restrito}. No Capítulo~\ref{cap:mwsp_cg}, abordamos o algoritmo exato de \emph{branch-and-price} proposto Sigurd e Zachariasen e uma implementação que propusemos para este algoritmo.
Discutimos os detalhes desta implementação que fizemos, como os pré-processamentos realizados, as heurísticas utilizadas para calcular limitantes primais e duais, assim como as decisões que afetam os nós numa árvore de B\&B, como a estratégia de ramificação e de busca do nó 
a ser explorado, e o processamento realizado em cada nó da árvore.
A implementação de Sigurd e Zachariasen não está disponível, e 
  os autores não comentam sobre a utilização de pré-processamentos e heurísticas
  para o cálculo de limitantes primais e duais.

No Capítulo~\ref{cap:experimentos}, apresentamos os experimentos realizados com as formulações lineares inteiras propostas no Capítulo~\ref{cap:mwstp} para o \textit{problema da árvore $t$-spanner de peso mínimo} (MWTSP).
Apresentamos também os experimentos computacionais para uma adaptação feita na
  implementação do algoritmo de \emph{branch-and-price} para que pudesse ser
utilizada para o MWTSP.
Também apresentamos os resultados computacionais dos testes relativos à implementação do algoritmo de \emph{branch-and-price} para o \textit{problema de
  $t$-spanner de peso mínimo} (MWSP). 

%a seguir, abordaremos alguns conceitos básicos que serão
% necessários ao longo da tese. Estes conceitos envolvem os seguintes tópicos: 
% grafos, poliedros, otimização combinatória e algoritmos de aproximação.
% O leitor que já estiver familiarizado com os tópicos poderá pular este
% capítulo.

No Capítulo~\ref{cap:conclusoes}, finalizamos a tese apresentando
  as considerações finais juntamente com algumas sugestões de trabalhos
  futuros.

%%% Local Variables:
%%% mode: latex
%%% coding: utf-8
%%% eval: (auto-fill-mode t)
%%% eval: (LaTeX-math-mode t)
%%% End:
