                                                                      % Arquivo LaTeX de exemplo de dissertação/tese a ser apresentados à CPG do IME-USP
% 
% Versão 5: Sex Mar  9 18:05:40 BRT 2012
%
% Criação: Jesús P. Mena-Chalco
% Revisão: Fabio Kon e Paulo Feofiloff
%  
% Obs: Leia previamente o texto do arquivo README.txt

\documentclass[12pt,twoside,a4paper]{book}

% ---------------------------------------------------------------------------- %
% Pacotes 
\usepackage[brazil]{babel}
\usepackage[T1]{fontenc}
%% \usepackage{ucs}
\usepackage[utf8x]{inputenc}
%% \usepackage[latin1]{inputenc}
\usepackage[pdftex]{graphicx}           % usamos arquivos pdf/png como figuras
\usepackage{setspace}                   % espaçamento flexível
\usepackage{indentfirst}                % indentação do primeiro parágrafo
\usepackage{makeidx}                    % índice remissivo
\usepackage[nottoc]{tocbibind}          % acrescentamos a bibliografia/indice/conteudo no Table of Contents
\usepackage{courier}                    % usa o Adobe Courier no lugar de Computer Modern Typewriter
\usepackage{type1cm}                    % fontes realmente escaláveis
\usepackage{listings}                   % para formatar código-fonte (ex. em Java)
\usepackage{titletoc}
%\usepackage[bf,small,compact]{titlesec} % cabeçalhos dos títulos: menores e compactos
\usepackage[fixlanguage]{babelbib}
\usepackage[font=small,format=plain,labelfont=bf,up,textfont=it,up]{caption}
\usepackage[usenames,svgnames,dvipsnames]{xcolor}
\usepackage[a4paper,top=2.54cm,bottom=2.0cm,left=2.0cm,right=2.54cm]{geometry} % margens
%\usepackage[pdftex,plainpages=false,pdfpagelabels,pagebackref,colorlinks=true,citecolor=black,linkcolor=black,urlcolor=black,filecolor=black,bookmarksopen=true]{hyperref} % links em preto

%---comentado por hugo---------
\usepackage[pdftex,plainpages=false,pdfpagelabels,pagebackref,colorlinks=true,citecolor=DarkGreen,linkcolor=NavyBlue,urlcolor=DarkRed,filecolor=green,bookmarksopen=true]{hyperref} % links coloridos
%-----------------------------

\usepackage[all]{hypcap}                % soluciona o problema com o hyperref e capitulos
\usepackage[square,sort,nonamebreak,comma]{natbib}  % citação bibliográfica alpha (alpha-ime.bst)
\fontsize{60}{62}\usefont{OT1}{cmr}{m}{n}{\selectfont}
%% \fontsize{60}{62}\usefont{OT1}{cmr}{m}{n}{\selectfont}

%-------------hugo-------------------------
%% \usepackage[ruled,linesnumbered,vlined,boxed]{algorithm2e}
\usepackage[portuguese,algosection,ruled,vlined,boxed]{algorithm2e}
\usepackage{amsfonts}
\usepackage{amsmath,calc} %para usar ocm LPs
\usepackage{mathtools,amssymb} %para usar com LPs
\usepackage{amsthm}
%\usepackage{subfigure}
\usepackage{subfig}
\usepackage[toc,page]{appendix}
\usepackage{mathrsfs} %mathscr
%enumerate package
\usepackage{enumitem}
%\usepackage{siunitx}
\usepackage{caption}
\usepackage{relsize} %\larger

%formatar tabela
\usepackage{makecell}

%comentar linha
\usepackage[normalem]{ulem}

%LP formulation------------
\usepackage{lpform} %LPs
\renewcommand{\lpforall}[1]{&& \forall #1}
\newlength{\LPlhbox} %LPs
\renewcommand{\lpsubjectto}{s.a}  %%%%%%%%%
%--------------------------

%tikz
\usepackage{tikz}
\usetikzlibrary{arrows, automata, arrows.meta, decorations.markings}

\usepackage{multirow}
\usepackage{booktabs}
\setlength\heavyrulewidth{0.05ex}
%separador decimal
%\usepackage[decimalsymbol=comma]{siunitx}
\usepackage{siunitx}
\sisetup{output-decimal-marker = {,}}
\sisetup{round-mode=places,round-precision=0}

%----------dados tese------------
\newcommand{\titulo}{Algoritmos exatos para problemas de spanner em grafos}
\newcommand{\autor}{Hugo Vinicius Vaz Braga}
%-------------------------------

\newcommand{\conjNaturais}{\mathbb{N}}
\newcommand{\conjInteiros}{\mathbb{Z}}
\newcommand{\conjInteirosPos}{\mathbb{Z}_{\ge 0}}

\newcommand{\Tscr}{\mathscr{T}}
\newcommand{\Pscr}{\mathscr{P}}
\newcommand{\Pcal}{\mathcal{P}}
\newcommand{\Scal}{\mathcal{S}}
\newcommand{\Ical}{\mathcal{I}}
\newcommand{\dist}{{\mathrm{dist}}}
\newcommand{\neigh}{{\mathrm{Neigh}}}
\newcommand{\arbset}{\Gamma}
\newcommand{\maxcam}{{\mathrm{maxCam}}}

\newcommand{\cluster}{\mathcal{S}}
%% \newcommand{\partition}{\mathcal{\tau}}
\newcommand{\partition}{\mathcal{T}}
\newcommand{\verneigh}{\Gamma^{ver}}
\newcommand{\comp}{\mathcal{C}}

\DeclarePairedDelimiter\ceil{\lceil}{\rceil}
\DeclarePairedDelimiter\floor{\lfloor}{\rfloor}

\newtheorem{definicao}{Definição}
\newtheorem{notacao}{Notação}
\newtheorem{teorema}{Teorema}[chapter]
\newtheorem{proposition}{Proposition}
\newtheorem{afirmacao}{Afirmação}[teorema]
\newtheorem{fato}{Fato}[chapter]
\newtheorem{lema}{Lema}[chapter]
\newtheorem{observacao}{Observação}
\newtheorem{proposicao}{Proposição}[chapter]
\newtheorem{corolario}{Corolário}
\newtheorem{conjectura}{Conjectura}
\newtheorem{problema}{Problema}
\newtheorem{exemplo}{Exemplo}
\newtheorem*{defi}{Definição}
\newtheorem*{propri}{Propriedades}
\newtheorem*{prop}{Proposição}
\newtheorem{ex}{Exercício}
\newtheorem{theorem}{Teorema}[section]
\newtheorem{conjecture}{Conjectura}
%comandos para novas letras/exppressões
\newcommand{\incid}{\mathcal{X}}
\newcommand{\incidY}{\mathcal{Y}}
\newcommand{\maxstretchfactor}{\mathcal{T}}
\newcommand{\espacoR}{\mathbb{R}}
\newcommand{\espacoRpos}{\mathbb{R}_{\ge 0}}
\newcommand{\espacoRn}{\mathbb{R}^n}
\newcommand{\espacoRp}{\mathbb{R}^p}
\newcommand{\espacoRq}{\mathbb{R}^q}
\newcommand{\espacoRdois}{\mathbb{R}^2_{\ge 0}}
\newcommand{\espacoRmn}{\mathbb{R}^{m \times n}}
\newcommand{\espacoRmp}{\mathbb{R}^{m \times p}}
\newcommand{\espacoRmq}{\mathbb{R}^{m \times q}}
\newcommand{\espacoRm}{\mathbb{R}^{m}}
\newcommand{\espacoXDef}{\mathbb{R}^{|E|}_{\ge 0}}
\newcommand{\espacoEDef}{\mathbb{R}^{|E|}}
\newcommand{\espacoX}{\mathbb{R}^{|E|}_{\ge 0}}
\newcommand{\espacoE}{\mathbb{R}^{|E|}}
\newcommand{\espacoY}{\mathbb{R}^{|E| \times |E|}_{\ge 0}}
\newcommand{\espacoYCG}{\mathbb{R}^{|\spanPath|}_{\ge 0}}
%% \newcommand{\espacoY}{\mathbb{R}^{|\spanPath| \times |E|}}
\newcommand{\espacoU}{\mathbb{R}^{|V| \times |V|}_{\ge 0}}
\newcommand{\PGtrestricoes}{\rm{(PL)}}
\newcommand{\FullPrimalPL}{\rm{Primal}}
\newcommand{\RMP}{\rm{RMP}}
\newcommand{\FullDualPL}{\rm{Dual}}
\newcommand{\FullDualTwoPL}{\rm{Dual\;2}}
\newcommand{\facetF}{\mathcal{F}}
\newcommand{\espacoNBinary}{\mathbb{B}^{n}}
\newcommand{\espacoXBinary}{\mathbb{B}^{|E|}}
\newcommand{\espacoYBinary}{\mathbb{B}^{|E| \times |E|}}
\newcommand{\espacoZBinary}{\mathbb{B}^{|V| \times |E| \times |E|}}
\newcommand{\espacoUuni}{\mathbb{R}^{|V|}}
\newcommand{\smallG}{\mbox{\larger[-1]$(G)$}}
\newcommand{\smallGf}{\mbox{\larger[-1]$(G-f)$}}
\newcommand{\smallGprime}{\mbox{\larger[-1]$(G')$}}
\newcommand{\smallGprimef}{\mbox{\larger[-1]$(G'-f)$}}
\newcommand{\smallGprimeG}{\mbox{\larger[-1]$(G',G)$}}
\newcommand{\smallW}{\mbox{\larger[-1]$(G[W])$}}
\newcommand{\smallWg}{\mbox{\larger[-1]$(G[W]-g)$}}
\newcommand{\smallWWbarf}{\mbox{\larger[-1]$(G[E(W) \cup E(\overline{W})+f])$}}
\newcommand{\smallWWbarfg}{\mbox{\larger[-1]$(G[E(W) \cup E(\overline{W})+f-g])$}}
\newcommand{\smallH}{\mbox{\larger[-1]$(H)$}}
\newcommand{\smallF}{\mbox{\larger[-1]$(G[F])$}}
\newcommand{\spanPath}{\mathcal{P}}
\newcommand{\spanPathPrime}{\mathcal{P}'}
\newcommand{\spanBridge}{\mathcal{B}}
\newcommand{\spanDel}{\mathcal{D}}
%% \newcommand{\Pathuv}{\spanPath_{u,v}^{t}}
\newcommand{\Pathuv}{\spanPath_{u,v}}
\newcommand{\PathuvPrime}{\spanPath_{u,v}'}
\newcommand{\Pathpq}{\spanPath_{p,q}^{t}}
\newcommand{\Path}{\spanPath^{t}}
\newcommand{\PathuvF}{\Pathuv\smallF}
\newcommand{\PathuvG}{\Pathuv\smallG}
\newcommand{\PathuvGprime}{\Pathuv\smallGprime}
\newcommand{\PathuvGf}{\Pathuv\smallGf}
\newcommand{\PathuvGprimef}{\Pathuv\smallGprimef}
\newcommand{\PathpqW}{\Pathpq\smallW}
\newcommand{\PathpqWg}{\Pathpq\smallWg}
\newcommand{\PathpqWWbarf}{\Pathpq\smallWWbarf}
\newcommand{\PathpqWWbarfg}{\Pathpq\smallWWbarfg}
\newcommand{\BridgeuvPrime}{\spanBridge_{u',v'}^{t}}
\newcommand{\BridgeuvPrimeG}{\spanBridge_{u',v'}^{t}(G)}
\newcommand{\Bridgeuv}{\spanBridge_{u,v}^{t}}
\newcommand{\BridgeuvG}{\spanBridge_{u,v}^{t}(G)}
\newcommand{\Bridge}{\spanBridge^{t}}
\newcommand{\BridgeG}{\spanBridge^{t}(G)}
\newcommand{\DelG}{\spanDel^{t}(G)}
\newcommand{\smallQBridge}{\mbox{\larger[-1]$(Q \cup \Bridge)$}}
\newcommand{\PathuvQBridge}{\Pathuv\smallQBridge}
\newcommand{\BridgeuvGprime}{\Bridgeuv\smallGprime}
\newcommand{\BridgeuvPrimeGf}{\spanBridge_{u',v'}^{t}\smallGf}
\newcommand{\BridgeuvGf}{\Bridgeuv\smallGf}
\newcommand{\BridgeGf}{\Bridge\smallGf}
\newcommand{\BridgeuvGprimeG}{\Bridgeuv\smallGprimeG}
\newcommand{\BridgeTwo}{\spanBridge_{2}^{t}}
\newcommand{\BridgeGprime}{\Bridge\smallGprime}
\newcommand{\BridgeTwoGprime}{\BridgeTwo\smallGprime}
\newcommand{\BridgeTwoGf}{\BridgeTwo\smallGf}
\newcommand{\setOfFixedSpan}{\mathcal{S}_{E_0, E_1}}

\newcommand{\opt}{\rm{opt}}

\newcommand{\arvorePeso}{\mathcal{T}^{w}}
\newcommand{\arvoreCusto}{\mathcal{T}^{c}}

\DeclareMathOperator*{\argmin}{arg\,min}

%evitar quebrar bloco
\newenvironment{absolutelynopagebreak}
  {\par\nobreak\vfil\penalty0\vfilneg
   \vtop\bgroup}
  {\par\xdef\tpd{\the\prevdepth}\egroup
   \prevdepth=\tpd}
%------------------------------------------


% ---------------------------------------------------------------------------- %
% Cabeçalhos similares ao TAOCP de Donald E. Knuth
\usepackage{fancyhdr}
\pagestyle{fancy}
\fancyhf{}
\renewcommand{\chaptermark}[1]{\markboth{\MakeUppercase{#1}}{}}
\renewcommand{\sectionmark}[1]{\markright{\MakeUppercase{#1}}{}}
\renewcommand{\headrulewidth}{0pt}

% ---------------------------------------------------------------------------- %
\graphicspath{{./figuras/}}             % caminho das figuras (recomendável)
\frenchspacing                          % arruma o espaço: id est (i.e.) e exempli gratia (e.g.) 
\urlstyle{same}                         % URL com o mesmo estilo do texto e não mono-spaced
\makeindex                              % para o índice remissivo
\raggedbottom                           % para não permitir espaços extra no texto
\fontsize{60}{62}\usefont{OT1}{cmr}{m}{n}{\selectfont}
\cleardoublepage
\normalsize

% ---------------------------------------------------------------------------- %
% Opções de listing usados para o código fonte
% Ref: http://en.wikibooks.org/wiki/LaTeX/Packages/Listings
\lstset{ %
language=Java,                  % choose the language of the code
basicstyle=\footnotesize,       % the size of the fonts that are used for the code
numbers=left,                   % where to put the line-numbers
numberstyle=\footnotesize,      % the size of the fonts that are used for the line-numbers
stepnumber=1,                   % the step between two line-numbers. If it's 1 each line will be numbered
numbersep=5pt,                  % how far the line-numbers are from the code
showspaces=false,               % show spaces adding particular underscores
showstringspaces=false,         % underline spaces within strings
showtabs=false,                 % show tabs within strings adding particular underscores
frame=single,	                % adds a frame around the code
framerule=0.6pt,
tabsize=2,	                    % sets default tabsize to 2 spaces
captionpos=b,                   % sets the caption-position to bottom
breaklines=true,                % sets automatic line breaking
breakatwhitespace=false,        % sets if automatic breaks should only happen at whitespace
escapeinside={\%*}{*)},         % if you want to add a comment within your code
backgroundcolor=\color[rgb]{1.0,1.0,1.0}, % choose the background color.
rulecolor=\color[rgb]{0.8,0.8,0.8},
extendedchars=true,
xleftmargin=10pt,
xrightmargin=10pt,
framexleftmargin=10pt,
framexrightmargin=10pt
}

% ---------------------------------------------------------------------------- %
% Corpo do texto
\begin{document}
\frontmatter 
% cabeçalho para as páginas das seções anteriores ao capítulo 1 (frontmatter)
\fancyhead[RO]{{\footnotesize\rightmark}\hspace{2em}\thepage}
\setcounter{tocdepth}{2}
\fancyhead[LE]{\thepage\hspace{2em}\footnotesize{\leftmark}}
\fancyhead[RE,LO]{}
\fancyhead[RO]{{\footnotesize\rightmark}\hspace{2em}\thepage}

\onehalfspacing  % espaçamento

% ---------------------------------------------------------------------------- %
% CAPA
% Nota: O título para as dissertações/teses do IME-USP devem caber em um 
% orifício de 10,7cm de largura x 6,0cm de altura que há na capa fornecida pela SPG.
\thispagestyle{empty}
\begin{center}
    \vspace*{2.3cm}
    \textbf{\Large{Algoritmos exatos para problemas \\ \vspace{-0.3cm} de \\ spanner em grafos}}\\
    
    \vspace*{1.2cm}
    \Large{\autor}
    
    \vskip 2cm
    \textsc{
    Tese apresentada\\[-0.25cm] 
    ao\\[-0.25cm]
    Instituto de Matemática e Estatística\\[-0.25cm]
    da\\[-0.25cm]
    Universidade de São Paulo\\[-0.25cm]
    para\\[-0.25cm]
    obtenção do título\\[-0.25cm]
    de\\[-0.25cm]
    Doutor em Ciência da Computação}
    
    \vskip 1.5cm
    Programa: Pós-Graduação em Ciência da Computação\\
    Orientador: Prof. Dr. rer. nat. Yoshiko Wakabayashi\\
    %% Coorientador: Prof. Dr. Nome do Coorientador

   	\vskip 1cm
    \normalsize{Durante o desenvolvimento deste trabalho o autor recebeu auxílio
    financeiro da FAPESP (Proc. 2013/22875-9)}
    
    \vskip 0.5cm
    \normalsize{São Paulo, fevereiro de 2019}
\end{center}

% ---------------------------------------------------------------------------- %
% Página de rosto (SÓ PARA A VERSÃO DEPOSITADA - ANTES DA DEFESA)
% Resolução CoPGr 5890 (20/12/2010)
%
% IMPORTANTE:
%   Coloque um '%' em todas as linhas
%   desta página antes de compilar a versão
%   final, corrigida, do trabalho
%
%
%% \newpage
%% \thispagestyle{empty}
%%     \begin{center}
%%         \vspace*{2.3 cm}
%%         \textbf{\Large{\titulo}}\\
%%         \vspace*{2 cm}
%%     \end{center}

%%     \vskip 2cm

%%     \begin{flushright}
%% 	Esta é a versão original da tese elaborada pelo\\
%% 	candidato \autor, tal como \\
%% 	submetida à Comissão Julgadora.
%%     \end{flushright}

%% \pagebreak


% ---------------------------------------------------------------------------- %
% Página de rosto (SÓ PARA A VERSÃO CORRIGIDA - APÓS DEFESA)
% Resolução CoPGr 5890 (20/12/2010)
%
% Nota: O título para as dissertações/teses do IME-USP devem caber em um 
% orifício de 10,7cm de largura x 6,0cm de altura que há na capa fornecida pela SPG.
%
% IMPORTANTE:
%   Coloque um '%' em todas as linhas desta
%   página antes de compilar a versão do trabalho que será entregue
%   à Comissão Julgadora antes da defesa
%
%
 \newpage
 \thispagestyle{empty}
     \begin{center}
         \vspace*{2.3 cm}
         %\textbf{\Large{\titulo}}\\
	\textbf{\Large{Algoritmos exatos para problemas \\ \vspace{-0.3cm} de \\ spanner em grafos}}\\
         \vspace*{2 cm}
     \end{center}

     \vskip 2cm

     \begin{flushright}
 	Esta versão da tese contém as correções e alterações sugeridas\\
 	pela Comissão Julgadora durante a defesa da versão original do trabalho,\\
 	realizada em 14/12/2018. Uma cópia da versão original está disponível no\\
 	Instituto de Matemática e Estatística da Universidade de São Paulo.

     \vskip 2cm

     \end{flushright}
     \vskip 4.2cm

     \begin{quote}
     \noindent Comissão Julgadora:
    
     \begin{itemize}
 		\item Profª. Drª. Yoshiko Wakabayashi (orientadora) - IME-USP
 		\item Prof. Dr.Carlos Eduardo Ferreira - IME-USP
 		\item Prof. Dr. Cláudio Nogueira de Meneses - UFABC
		\item Profª. Drª. Karla Roberta Pereira Sampaio Lima - EACH-USP
		\item Prof. Dr. Eduardo Cândido Xavier - UNICAMP
     \end{itemize}
      
     \end{quote}
 \pagebreak


\pagenumbering{roman}     % começamos a numerar 

% ---------------------------------------------------------------------------- %
% Agradecimentos:
% Se o candidato não quer fazer agradecimentos, deve simplesmente eliminar esta página 
\chapter*{Agradecimentos}
Começo fazendo um agradecimento em especial à minha orientadora,
Profª Yoshiko Wakabayashi, por todos os ensinamentos e dedicação que teve
para comigo. Ter tido a oportunidade de trabalhar com uma pessoa tão
qualificada e premiada, e que aceitou orientar um estudante
iniciante na área, gera um sentimento de gratidão 
desde o momento em que você aceitou ser minha orientadora. 
Certamente Yoshiko foi fundamental para a realização deste trabalho.


Agradeço à minha família, em especial aos meus pais, por todo o apoio ao
longo deste doutorado. Em momentos importantes, como na reta final, a
presença de vocês foi importante na continuidade deste trabalho. É claro que
não posso deixar de mencionar meus irmãos Júnior, Clovis e Marcelle, além
da minha avó Eufélia.

Agradeço aos colegas de laboratório, em especial ao Victor, Gabriel e Tiago.
Gostaria de agradecer também aos professores com os quais tive a possibilidade
de iniciar os estudos na área de otimização combinatória.

Agradeço em especial ao Prof. Alexandre Freire pela orientação e suporte
na implementação do algoritmo de \emph{branch-and-price}. A falta de
experiência na área de geração de colunas 
foi minimizada pela ajuda do professor Alexandre. Outro
agradecimento em especial ao Prof. Manoel Campêlo pela contribuição na
elaboração dos algoritmos exatos para o problema de árvore $t$-spanner.

Agradeço aos amigos de estatística Douglas,
Karina, Guilherme, Aline e Rodrigo (goiano) por terem me recebido tão bem em
São Paulo.

Agradeço à Fundação de Amparo à Pesquisa do Estado de São Paulo (FAPESP)
pelo financiamento deste trabalho por meio do processo 2013/22875-9. Graças
ao suporte da \mbox{FAPESP}, pude me dedicar de forma exclusiva
à elaboração deste trabalho.

% ---------------------------------------------------------------------------- %
% Resumo
\chapter*{Resumo}

\noindent BRAGA, H. \textbf{\titulo}. 
2018. 96 f.
Tese (Doutorado) - Instituto de Matemática e Estatística,
Universidade de São Paulo, São Paulo, 2018.
\\

Seja $(G,w,t)$ uma tripla formada por um grafo conexo $G = (V,E)$, uma
função peso não-negativa $w$ definida em $E$ e um número real $t \ge 1$,
chamado de \emph{fator de dilatação}.  Um \emph{$t$-spanner} de $G$ é um
subgrafo gerador $H$ de $G$ tal que para cada par de vértices $u,v$, a
distância entre $u$ e $v$ em $H$ é no máximo $t$ vezes a distância
entre $u$ e $v$ em $G$.  Quando $H$ é uma árvore, dizemos que $H$ é
uma \emph{árvore $t$-spanner}.  Nesta tese focamos o \emph{problema da
  árvore $t$-spanner de peso mínimo} (cuja sigla em inglês é MWTSP),
definido a seguir. Dada uma tripla $(G,w,t)$, encontrar uma árvore
$t$-spanner em $G$ de peso mínimo. É sabido que MWTSP é NP-difícil
para cada $t > 1$ fixo. Propomos duas formulações lineares inteiras
para MWTSP, baseadas em arborescência, de tamanho polinomial no tamanho de $G$.  A
formulação que possui variáveis representando distâncias entres os
pares de vértices apresentou resultados melhores na prática.

Também abordamos o \emph{problema de $t$-spanner de peso mínimo} (cuja
sigla em inglês é MWSP), cuja entrada é a mesma do MWTSP e cujo
objetivo é encontrar um $t$-spanner de peso mínimo. Mesmo para grafos
com peso unitário, MWSP é NP-difícil para cada $t \ge 2$ fixo.
Tratamos este problema de duas formas. Propomos uma formulação linear
inteira para o MWSP que possui um número exponencial de restrições, mas
cujo problema da separação --- para o programa linear relaxado
correspondente --- é polinomial no tamanho de $G$. Apresentamos também uma implementação
de um algoritmo de \emph{branch-and-price} para o MWSP baseado numa
formulação linear inteira proposta por Sigurd e Zachariasen (2004).
Exibimos resultados de experimentos realizados com
as duas formulações para o MWTSP e também com o algoritmo de
\emph{branch-and-price} para o MWSP.
\\

\noindent \textbf{Palavras-chave:} grafo, spanner, geração de coluna, algoritmo exato.

% ---------------------------------------------------------------------------- %
% Abstract
\chapter*{Abstract}
\noindent BRAGA, H. \textbf{Exact algorithms for spanner problems in graphs}. 
2018. 96 f.
Tese (Doutorado) - Instituto de Matemática e Estatística,
Universidade de São Paulo, São Paulo, 2018.
\\


Let $(G,w,t)$ be a triplet consisting of a connected graph $G = (V,E)$ with
a nonnegative weight function $w$ defined on $E$, and a real number $t \ge 1$.
A \emph{$t$-spanner} of $G$ is a spanning subgraph $H$ of $G$ such that for each pair
of vertices $u,v$, the distance between $u$ and $v$ in $H$ is at most $t$ times
the distance between $u$ and $v$ in $G$. If $H$ is a tree then we call it a
\emph{tree $t$-spanner} of $G$.  We address the
\emph{Minimum Weight Tree Spanner Problem} (MWTSP), defined as follows.
Given a triplet $(G,w,t)$, find a
minimum weight tree $t$-spanner in $G$. It is known that MWTSP is NP-hard
for every fixed $t > 1$.  We propose two ILP formulations for MWTSP,
based on arborescences, of polynomial size in the size of $G$.  The formulation that has
variables representing distances between pairs of vertices has shown
to be better in practice. 

We also address the \emph{Minimum Weight t-Spanner Problem} (MWSP)
that has the same input as MWTSP and seeks for a minimum weight
$t$-spanner in $G$.  Even for unweighted graphs, it is known that MWSP
is NP-hard for every fixed $t \ge 2$. We approach this problem in two
ways. We propose an ILP formulation for MWSP that has an an
exponential number of restrictions but we show that the separation
problem --- for the corresponding relaxed linear program --- can be
solved in polynomial time in the size of $G$.  We also present a branch-and-price
algorithm for MWSP based on an ILP formulation proposed by Sigurd and
Zachariasen (2004).  We show results on the computational experiments
with both formulations for MWTSP and also with the branch-and-price
algorithm that we implemented for MWSP.
\\

\noindent \textbf{Keywords:} graph, spanner, column generation, exact algorithm.

% ---------------------------------------------------------------------------- %
% Sumário
\tableofcontents    % imprime o sumário

% ---------------------------------------------------------------------------- %
\input cap-abreviaturas
% ---------------------------------------------------------------------------- %
\input cap-simbolos
% ---------------------------------------------------------------------------- %
% Listas de figuras e tabelas criadas automaticamente
\listoffigures            
\listoftables
\listofalgorithms
\addcontentsline{toc}{chapter}{Lista de Algoritmos}

% ---------------------------------------------------------------------------- %
% Capítulos do trabalho
\mainmatter

% cabeçalho para as páginas de todos os capítulos
\fancyhead[RE,LO]{\thesection}

%% \singlespacing              % espaçamento simples
\onehalfspacing            % espaçamento um e meio

%% \input cap-conceitos         % associado ao arquivo: 'cap-conceitos.tex'
\input cap1-introducao        % associado ao arquivo: 'cap-introducao.tex'
\input cap2-preliminares
 \input cap3-historico
 \input cap4-MWTSP
 \input cap5-MWSP
 \input cap6-MWSP_CG
 \input cap7-experimentos
 \input cap8-conclusao

% cabeçalho para os apêndices
\renewcommand{\chaptermark}[1]{\markboth{\MakeUppercase{\appendixname\ \thechapter}} {\MakeUppercase{#1}} }
\fancyhead[RE,LO]{}
\appendix

%% \include{ape-conjuntos}      % associado ao arquivo: 'ape-conjuntos.tex'
%% \include{ape-grafos}      

% ---------------------------------------------------------------------------- %
% Bibliografia
\backmatter \singlespacing   % espaçamento simples
\bibliographystyle{alpha-ime}% citação bibliográfica alpha
%% \begingroup
%% \inputencoding{latin1}
%% \bibliography{bibliografia}  % associado ao arquivo: 'bibliografia.bib'
%% \endgroup
\bibliography{bibliografia}  % associado ao arquivo: 'bibliografia.bib'

% ---------------------------------------------------------------------------- %
% Índice remissivo
%% \index{TBP|see{periodicidade região codificante}}
%% \index{DSP|see{processamento digital de sinais}}
%% \index{STFT|see{transformada de Fourier de tempo reduzido}}
%% \index{DFT|see{transformada discreta de Fourier}}
%% \index{Fourier!transformada|see{transformada de Fourier}}

%% \printindex   % imprime o índice remissivo no documento 

\end{document}

%%% Local Variables:
%%% mode: latex
%%% TeX-master: "latexmk"
%%% End:
