%!TEX root = main.tex

%% ------------------------------------------------------------------------- %%
\chapter{Considerações finais}
\label{cap:conclusoes}
Neste capítulo tecemos algumas considerações sobre nossas
contribuições para os problemas que focamos, e mencionamos algumas
linhas ou questões que poderiam ser investigadas.

No Capítulo~\ref{cap:mwstp}, propusemos duas formulações lineares
inteiras para o problema da árvore $t$-spanner de peso mínimmo
(MWTSP). Estas são as primeiras formulações lineares específicas para
este problema.  Elas são baseadas numa ideia interessante, que é a de
modelar via arborescências (requerendo lidar com digrafos), que
facilita a identificação dos caminhos entre cada par de vértices na
árvore (que resulta da ``sobreposição das arborescências'').  Ambas as
formulações são de tamanho polinomial no tamanho de $G$. Elas se diferem quanto ao uso
ou não de variáveis (rótulos) para representar as distâncias entre os
pares de vértices. Os experimentos computacionais mostraram que a
formulação com rótulos tem um desempenho melhor (do que a formulação
sem rótulos explícitos). Essa foi uma descoberta que os experimentos
computacionais forneceram, mostrando que, apesar de usar mais
variáveis, em geral, vale a pena usar a formulação com rótulos.  Neste
caso também seria interessante encontrar inequações válidas para
fortalecer a formulação linear correspondente.

% Denominamos por algoritmo \emph{com rótulos} aquele que possui
% rótulos para representar as distâncias entre os pares de vértices. O outro
% algoritmo foi denominado por \emph{sem rótulos}. Como observamos na
% Tabela~\ref{tab:conc_alg_exato}, o algoritmo CR foi melhor na maioria dos
% cenários analisados.

Na Seção~\ref{sec:mwsp_preprocessamento}, abordamos uma
heurística para o MWTSP, mas nos experimentos realizados a
heurística foi desabilitada. Seria interessante analisar em
quais cenários ela é útil.

No Capítulo~\ref{cap:mwsp} apresentamos uma formulação linear inteira
para o problema de $t$-spanner de peso mínimo (MWSP), sobre a qual
nada mencionamos em termos de testes computacionais ou outros
aspectos.  Nesta formulação, não existe variável associada a caminhos,
uma ideia que seria natural, já que restrições sobre comprimentos de
caminhos têm que ser impostas. As variáveis usadas, além daquelas
associadas às arestas, são indexadas por pares formados por arestas e
cortes do grafo.  Estas últimas levam a uma formulação com um número
exponencial de restrições, cujo problema de separação --- considerando
o programa linear relaxado correspondente --- pode ser resolvido em
tempo polinomial (no tamanho de $V$). Este fato, a princípio promissor, acabou não se
mostrando muito útil: alguns experimentos iniciais mostraram que a
formulação não apresentava um bom desempenho. Nesta direção, uma das
ideias a ser explorada mais sistematicamente seria a de tentar
descobrir inequações válidas (preferencialmente facetas) para
fortalecer o programa relaxado.  Para isso, poderíamos utilizar
\emph{softwares} como Porta~\cite{Porta2018},
Panda~\cite{LorwaldR2015} e Skeleton~\cite{Zolotykh2012}. Embora não
tenhamos insistido muito nesta direção, pois focamos outras
alternativas, talvez seja uma direção que vale a pena ser considerada.

% Podemos tentar também descobrir inequações válidas que não necessariamente
% são facetas, como fizemos para uma das formulações lineares inteiras
% propostas para um outro problema de spanner (veja Seção~\ref{sec:ineq_valida}).

No Capítulo~\ref{cap:mwsp_cg} apresentamos uma implementação para um
algoritmo de \emph{branch-and-price} para este mesmo problema (o
MWSP).  O algoritmo foi proposto por Sigurd e
Zachariasen~\cite{SigurdZ2004}, mas esses autores não disponibilizaram
o código. Na implementação que apresentamos aqui utilizamos pré-processamentos e
heurísticas para o cálculo de limitantes primais e duais.  Para o RMP
associado ao programa linear considerado, as restrições
(\ref{rmp:relate_vars}) do RMP podem ser removidas para encontrar mais
rapidamente uma solução viável. Como mencionado por Sigurd e
Zachariasen~\cite{SigurdZ2004}, o teste para saber se a solução
encontrada viola uma destas restrições é rápido. Os autores ainda
observaram que somente um pequeno subconjunto das inequações precisou
ser adicionado ao RMP, acelerando bastante a resolução do
problema. Poderíamos fazer o mesmo, possivelmente conseguindo
resultados melhores do que os apresentados no
Capítulo~\ref{cap:experimentos}. Um outro aspecto a ser explorado
seria investigar outras heurísticas tanto para o cálculo de limitantes
primais como de limitantes duais.

% \section{Problema da árvore $t$-spanner}
% \label{sec:conc_mwtsp}

Nos experimentos realizados, os grafos de entrada são gerados de maneira
aleatória, e as arestas possuem um dos três tipos de pesos adotados
(veja Seção~\ref{sec:parametros}). Levando em consideração que spanners
são adotados em oráculos de distâncias (veja a Seção~\ref{sec:aplicacoes}),
utilizar outros grafos de entradas como mapas que representam distâncias
entre cidades seria bastante útil. Um outro cenário interessante que
configura um bom exemplo de entrada são as rotas utilizadas
por robôs, visto que spanners são bastante utilizados na área de
planejadores de roteiros em robótica.

% Assim como comentamos na seção anterior, podemos tentar encontrar
% inequações válidas para as formulações, como fizemos para o algoritmo
% CR (veja Seção~\ref{sec:ineq_valida}).

Por fim, realizar experimentos com um número maior de instâncias
deixaria mais claro o comportamento das curvas apresentadas nos
gráficos. Em alguns experimentos, percebemos que poderíamos fazer
testes para valores maiores de alguns parâmetros. Conhecer os limites
em termos de variação de valores dos parâmetros traria uma informação
relevante a respeito das implementações.

Esperamos em breve disponibilizar os códigos das implementações que
fizemos, para que pessoas interessadas nos problemas focados possam se
beneficiar disso. Ao iniciar os nossos estudos, esbarramos na
dificuldade de obter dados e de poder comparar nossas implementações
com outras. Acreditamos assim que este trabalho poderá contribuir
na direção de facilitar outros estudos sobre os problemas aqui
tratados. 

%------------------------------------------------------
%% \section{Considerações finais} 

%% Texto texto texto texto texto texto texto texto texto texto texto texto texto
%% texto texto texto texto texto texto texto texto texto texto texto texto texto
%% texto texto texto texto texto texto. 

%% %------------------------------------------------------
%% \section{Sugestões para pesquisas futuras} 

%% Texto texto texto texto texto texto texto texto texto texto texto texto texto
%% texto texto texto texto texto texto texto texto texto texto texto texto texto
%% texto texto texto texto texto texto.

