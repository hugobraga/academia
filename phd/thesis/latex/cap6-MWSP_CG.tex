%!TEX root = main.tex

\chapter{Um algoritmo de \emph{branch-and-price} para o
  problema de  $t$-spanner de peso mínimo}
\label{cap:mwsp_cg}

% valido para todos os algoritmos
\SetInd{0.5em}{0.6em}
\SetKwInOut{Input}{Input}
\SetKwInOut{Output}{Output}

Neste capítulo abordamos o problema de $t$-spanner de peso mínimo
(MWSP), focando na implementação que fizemos de um algoritmo de
\emph{branch-and-price} para o MWSP.
Apresentamos na Seção~\ref{sec:mwsp_cg} o
algoritmo proposto por Sigurd e Zachariasen~\cite{SigurdZ2004}, no
qual baseamos a nossa implementação. Na seção seguinte apresentamos os
detalhes da implementação, mencionando primeiramente os
pré-processamentos realizados. Em seguida, abordamos as heurísticas
implementadas para calcular limitantes primais e duais.
Por fim, discutimos estratégias de ramificação e de busca do nó na
árvore de B\&B, finalizando com detalhes sobre o processamento dos nós
desta árvore. No artigo de Sigurd e Zachariasen, os autores não comentam 
 aspectos sobre adoção de pré-processamentos e heurísticas para o cálculo de
  limitantes.

%% \section{Contexto}
%% No trabalho de Sigurd e Zachariasen~\cite{SigurdZ2004}, os autores estavam
%% interessados em avaliar a performance da heurística gulosa proposta por
%% Alth\"{o}fer et al~\cite{AlthoferDDJS1993}. Para isso, eles propuseram
%% um algoritmo exato para o MWSP baseado em geração de colunas. A heurística
%% alvo de comparação está descrita no Algoritmo \ref{alg:greedy_spanner}.

%% \SetInd{0.5em}{0.6em}
%% \SetKwInOut{Input}{Input}
%% \SetKwInOut{Output}{Output}
%% \begin{algorithm}[H]
%%   \SetAlgoLined
%%   \Input{$G=(V, E)$, real $t > 1$}
%%   \Output{$t$-spanner $G'$ de $G$ com peso pequeno}
%%   \BlankLine
%%   Ordene $E$ pelo peso de maneira não decrescente\;
%%   $G' \gets (V,E' = \emptyset)$\;
%%   \ForEach{$e=uv \in E$}{
%%     \lIf{$t \cdot w(e) < \dist_{G'}(u,v)$}{$E'$ $\gets$ $E' \cup e$}
%%   }
%%   \caption{Algoritmo Guloso} 
%%   \label{alg:greedy_spanner}
%% \end{algorithm}

\section{Geração de colunas para resolver formulação relaxada}
\label{sec:mwsp_cg}
Em 2004, Sigurd e Zachariasen~\cite{SigurdZ2004} apresentaram um
algoritmo exato para uma generalização do MWSP em que a restrição de
spanner deve ser satisfeita por um subconjunto de pares de
vértices. Nesta tese, consideramos a versão clássica (em vez da versão
generalizada), na qual todos os pares de vértices devem satisfazer a
restrição de spanner.
Vamos apresentar nesta seção o algoritmo exato proposto por
Sigurd e Zachariasen.

% Como apresentado no Capítulo~\ref{cap:mwsp}, para um grafo $G=(V,E)$,
% com peso \hbox{$w: E \to \mathbb{R}^+$,} e um real $t \ge 1$, o
% poliedro do problema de grafos $t$-spanner associado a $G,w,t$, é
% definido como

Como apresentado no Capítulo~\ref{cap:mwsp}, o poliedro dos
$t$-spanners de um grafo $G=(V,E)$, onde $t\ge 1$ é um número real, é
definido como
\begin{equation*}
\begin{split}
P_{span}(G,t) := \text{conv}\{\incid^{F} \in \espacoE\; |\;
\text{$G[F]$ é um $t$-spanner de $G$\}}. 
\end{split}
\end{equation*}

A formulação proposta por Sigurd e Zachariasen~\cite{SigurdZ2004} para
o MWSP$(G,w,t)$ é a seguinte. Estamos interessados em conjuntos $F\subseteq E$
tais que $\incid^{F} \in P_{span}(G,t)$.  Conforme vimos no início do
Capítulo~\ref{cap:conceitos_historico-v2}, para $e=uv \in E$, denotamos por
$\Pathuv$ a coleção de caminhos $t$-spanner de $u$ a $v$ em~$G$. Seja
$\spanPath = \bigcup\limits_{uv \in E} \Pathuv$.  

A formulação linear faz uso de dois tipos de variáveis de
decisão~$0/1$, como segue.  (a)~Variável $x \in \espacoE$ com o seguinte
significado: para cada aresta $e \in E$, $x_e = 1$ se e só se $e$
faz parte da solução $F$. (b)~Variável $y \in \espacoYCG$ com o
seguinte significado: para cada aresta $e=uv \in E$ e para cada
caminho $p \in \Pathuv$, $y_p = 1$ se e só se $p$ conecta $u$ a $v$ na
solução.

Considere a seguinte formulação linear~$0/1$ (proposta por Sigurd e
Zachariasen). Nessa formulação, para $e \in E$, $p \in \spanPath$,
considere que $\delta^e_p = 1$ se a aresta $e$ pertence ao caminho
$p$, e $\delta^e_p = 0$, em caso contrário. 
%
\begin{lpformulation}[\FullPrimalPL]
\lpobj*{min}{\sum_{e\in E} w_ex_e}
\lpeq[res_mwsp_cg:relate_vars]{\sum_{p \in \Pathuv} \delta^e_p y_p \le x_e}{e \in E,\, \forall uv \in E}
\lpeq[res_mwsp_cg:caminho]{\sum_{p \in \Pathuv} y_p \ge 1}{uv \in E}
\lpeq[res_mwsp_cg:int_x]{x_e \in \{0,1\}}{e \in E}
\lpeq[res_mwsp_cg:int_y]{y_p \in \{0,1\}}{p \in \spanPath}
\end{lpformulation}
%% $P(G,t)= \{(x, y): x \in \espacoE, y \in \espacoYCG\; |\; (x,y) \text{ satisfaz PL } \FullPrimalPL\}.$

A inequação (\ref{res_mwsp_cg:caminho}) diz que para cada par de
vértices $u,v$ que define uma aresta, deve existir pelo menos um
caminho $t$-spanner entre $u$ e $v$ na solução.  Para cada aresta $e$
que faça parte de um caminho $t$-spanner selecionado, a
inequação~(\ref{res_mwsp_cg:relate_vars}) força que $e$ pertença à
solução.

Sigurd e Zachariasen usam o método de geração de colunas para resolver
a relaxação do programa $\FullPrimalPL$ (que resulta trocando-se as
restrições de integralidade (\ref{res_mwsp_cg:int_x}) e
(\ref{res_mwsp_cg:int_y}) por restrições de não negatividade).  Em
decorrência do número exponencial de variáveis $y_p$, considera-se uma
versão reduzida da relaxação do programa $\FullPrimalPL$, denotada por RMP
(\emph{Reduced Master Problem}).
%% conhecida como \emph{problema mestre reduzido}
%% (\emph{Reduced Master Problem} - RMP).

Para cada $uv \in E$, considere $\PathuvPrime \subseteq \Pathuv$ e
$\spanPathPrime = \bigcup\limits_{uv \in E} \PathuvPrime$. O RMP do programa $\FullPrimalPL$ é definido da seguinte forma:
%
\begin{lpformulation}[\RMP]
\lpobj*{min}{\sum_{e\in E} w_ex_e}
\lpeq[rmp:relate_vars]{\sum_{p \in \PathuvPrime} \delta^e_p y_p \le x_e}{e \in E,\, \forall uv \in E}
\lpeq[rmp:caminho]{\sum_{p \in \PathuvPrime} y_p \ge 1}{uv \in E}
\lpeq[]{x_e \ge 0}{e \in E}
\lpeq[]{y_p \ge 0}{p \in \spanPathPrime}
\end{lpformulation}

Para resolver o programa linear $\RMP$, basta que para cada $e=uv \in E$,
$\PathuvPrime$ contenha inicialmente apenas um caminho. Para cada
iteração, novas colunas (que representam novos caminhos em
$\spanPathPrime$) vão sendo adicionadas ao programa linear caracterizando a
abordagem conhecida como geração de colunas. Para cada $uv, e \in E$,
sejam $\pi^{u,v}_{e}$ e $\sigma_{u,v}$ as variáveis duais associadas
às inequações (\ref{res_mwsp_cg:relate_vars}) e
(\ref{res_mwsp_cg:caminho}) respectivamente. O dual da relaxação do
programa $\FullPrimalPL$ é descrito pelo seguinte PL:
%
\begin{lpformulation}[\FullDualPL]
\lpobj*{max}{\sum_{uv \in E} \sigma_{u,v}}
\lpeq[dual:ineq1]{\sum_{uv \in E} \pi^{u,v}_{e} \le w_e}{e \in E}
\lpeq[dual:ineq2]{\sum_{e \in E} \delta^{e}_{p}\pi^{u,v}_{e} - \sigma_{u,v} \ge 0}{uv \in E,\, \forall p \in \Pathuv}
\lpeq[]{\pi^{u,v}_{e} \ge 0}{e \in E,\, \forall uv \in E}
\lpeq[]{\sigma_{u,v} \ge 0}{uv \in E}
\end{lpformulation}

A resolução do RMP serve para fornecer soluções (duais) para o
programa \FullDualPL.  Seja $(x',y')$ uma solução ótima do RMP e seja
$(\pi', \sigma')$ a solução dual associada. Se $(\pi', \sigma')$ for
uma solução viável para o programa \FullDualPL, então $(x',y')$ é uma
solução ótima para $\FullPrimalPL$ (a ser explicado a seguir). Considere a
seguinte expressão:

\begin{align}
  \label{def:reduced_cost}
  c^{\pi, \sigma}_{p} = \sum_{e \in E} \delta^{e}_{p}\pi^{u,v}_{e} - \sigma_{u,v}, \:\; \forall uv \in E, \: \forall p \in \Pathuv. 
\end{align}

O valor $c^{\pi, \sigma}_{p}$ é denominado \emph{custo reduzido}
(\emph{reduced cost}) do caminho $p$ com relação às variáveis
$\pi$ e $\sigma$.
Caso a solução $(\pi', \sigma')$ seja uma solução viável para o programa \FullDualPL,
significa que para cada $uv \in E$ e para cada $p \in \Pathuv$, o custo
reduzido associado é maior ou igual a zero. Neste caso, temos que a solução $(x',y')$
é ótima (veja o Teorema~3.1 em Bertsimas e Tsitsiklis~\cite{BertsimasT1997}). Caso contrário, 
%% Caso a solução $(\pi', \sigma')$ não seja uma solução viável para o PL
%% \FullDualPL, então
para algum par $uv \in E$ e $p \in \spanPath$, a inequação (\ref{dual:ineq2})
é violada, ou seja, $c^{\pi', \sigma'}_{p} < 0$. A inequação (\ref{dual:ineq1})
não é violada pois, para cada $e \in E$, a inequação (\ref{dual:ineq1})
está relacionada à coluna da variável $x_e$ no programa 
\FullPrimalPL, e todas estas colunas já estão presentes no RMP.

Para um par $uv \in E$ e $p \in \spanPath$ tal que a inequação
(\ref{dual:ineq2}) é violada, devemos adicionar o caminho $p$ (entre
os vértices $u$ e $v$) ao RMP.  Em cada iteração da geração de
colunas, para cada $uv \in E$, queremos encontrar um caminho
$t$-spanner entre $u$ e $v$ com o menor valor de custo reduzido. Para
isso, devemos resolver

\begin{align}
  \label{def:pricing}
  \min_{p \in \Pathuv}\sum_{e \in E} \delta^{e}_{p}\pi^{u,v}_{e} - \sigma_{u,v}.
\end{align}
Se o valor da solução do problema (\ref{def:pricing}) for menor do que
zero, então encontramos o caminho a ser inserido no RMP. Caso
contrário, a geração de colunas é finalizada.  O problema representado
pela expressão (\ref{def:pricing}) é conhecido como o problema do
\emph{pricing}. Para cada $uv \in E$, $\sigma_{u,v}$ é constante e o
\emph{pricing} consiste em encontrar um caminho $t$-spanner entre $u$
e $v$ de menor peso, sendo que o peso das arestas é dado pelos valores
da variável dual $\pi$. Este problema é conhecido como \emph{problema
  do caminho mínimo restrito} (\emph{Constrained Shortest Path
  Problem} - CSPP).

\subsection{Problema do caminho mínimo restrito}
O problema do caminho mínimo restrito (CSPP) é definido da seguinte forma:
dado um grafo $G = (V,E)$, uma função peso $w: E \to \conjInteirosPos$, uma
função custo $c: E \to \conjInteirosPos$, vértices $s,d \in V$, e uma
constante $B$ denominada \emph{limite de recurso}, o objetivo é encontrar
um caminho de custo mínimo entre $s$ e $d$ com relação à função custo $c$,
cuja soma dos pesos das arestas seja menor ou igual a $B$. A versão de decisão
deste problema, onde se considera não apenas um limitante para o peso mas
também um limitante para o custo, é NP-Completo
(veja problema ND30 em~\cite{GareyJ1990}).

Para resolver o CSPP, Sigurd e Zachariasen~\cite{SigurdZ2004} adotaram
o algoritmo baseado em rótulos proposto por
Desrochers e Soumis~\cite{DesrochersS1988}. Neste algoritmo, caminhos a
partir do vértice $s$ vão sendo construídos e incrementados até atingir
o vértice $d$. Um caminho $p_i$ é representado pelo rótulo
$(p_i, v_i, w_i, c_i)$, onde $n_i$ representa o vértice final do caminho,
$w_i$ e $c_i$ representam o peso e o custo total respectivamente de $p_i$.
Dizemos que um caminho $p_i$ \textit{domina} um caminho $p_j$ se $v_i = v_j$ e
$w_i \le w_j$ e $c_i \le c_j$.
%% ??Ao longo da execução do algoritmo, que
%% funciona em rodadas, os rótulos dominados vão sendo removidos.

O algoritmo funciona em rodadas, e em cada rodada é escolhido o rótulo
$p_i$ com menor custo. O caminho $p_i$ é estendido para
cada vizinho de $v_i$, dando origem a novos rótulos. Cada rótulo dominado,
novo ou antigo, deve ser eliminado.
Seja $\dist_{G}(u,v,w)$ a distância entre  $u$ e $v$ em $G$ medida pela
função $w$. De maneira análoga, defina $\dist_{G}(u,v,c)$. 
Para cada novo caminho criado $p_j$, se $w_j + \dist_{G}(v_j,d,w) > B$,
 descartamos $p_j$. Seja $M$ uma constante que é um limitante para o custo máximo
dos caminhos. Além da verificação anterior, se
$c_j + \dist_{G}(v_j,d,c) > M$, o caminho $p_j$ é descartado.
Para calcular as distâncias entre os vértices e o vértice destino~$d$,
árvores de caminhos mínimos são construídas. 
%% ??Para calcular as distâncias a partir do vértice $d$,
Vamos chamar de $\arvorePeso$ a árvore
de caminhos de peso mínimo de $G$ a partir de $d$ e de $\arvoreCusto$ a árvore
de caminhos de custo mínimo de $G$ a partir de $d$.

%% Para cada $v \in V$,
%% seja $\neigh_{G}(v)$ o conjunto dos vértices vizinhos de $v$ em $G$.
Para cada iteração da geração de colunas, o Algoritmo~\ref{alg:cspp}
é executado para cada par $e=uv \in E$.
O peso (resp. custo) das arestas é dado
pela função $w$ associada a $G$ (resp. $\pi^{u,v}$).
Os limitantes
$M$ e $B$ possuem valores $\sigma^{u,v}$ e $t \cdot w_e$ respectivamente.
Os vértices de origem e destino são $u$ e $v$ respectivamente. 


O Algoritmo~\ref{alg:cspp} é utilizado para resolver o CSPP.

%\medskip

%% \SetInd{0.5em}{0.6em}
\begin{algorithm}
  \SetAlgoLined
  
  \textbf{Entrada:} {$G=(V, E)$, $w: E \to \espacoRpos$, $c: E \to \espacoRpos$, número real $B$, número real $M$, $s \in V$, $d \in V$}\;
  
  \textbf{Saída:} {caminho $p_f$ entre $s$ e $d$ de menor custo satisfazendo a restrição de peso máximo $B$ e de custo máximo $M$}\;
  
  $p_f \gets null$\;
  Calcula as árvores de caminhos de peso e custo mínimos $\arvorePeso$ e $\arvoreCusto$\;
  $H \gets \{(\{\},s,0,0)\}$\;
  \While{$H \neq \emptyset$}{
    Escolha o rótulo $(p_{\alpha},v_{\alpha},w_{\alpha},c_{\alpha})$ mais barato (em termos de custo) no conjunto $H$\;
    \If{$v_{\alpha} = d$}{
      $p_f \gets p_{\alpha}$\;
      \Return $p_f$.
    }
    \ForEach{$v_i \in N_{v_{\alpha}}(G)$}{
      Seja $e_i \gets v_{\alpha}v_i$.
      Crie o novo rótulo $(p_{\alpha} \cup \{e_i\}, v_i, w_{\alpha} + w_{e_i}, c_{\alpha} + c_{e_i})$\;
      Descarte todos os novos rótulos tal que $w_{\alpha} + w_{e_i} + \arvorePeso(v_i) > B$ ou
      $c_{\alpha} + c_{e_i} + \arvoreCusto(v_i) > M$. Para os demais, armazene em $H$\;
      Descarte (de $H$) rótulos dominados\;
    }
    \Return $p_f$.
  }
  \caption{CSPP} 
  \label{alg:cspp}
\end{algorithm}

%% O algoritmo \ref{alg:cspp} é executado para cada par $e=uv \in E$,
%% $CSPP(G,w, \pi^{u,v}, t \cdot w_{e}, \sigma^{u,v}, u, v)$.

\section{Detalhes da implementação}
\label{sec:implementacao}
%% Nós estamos utilizando como referência~\cite{BarnhartJNSV1998,NemhauserW1999,ConfortiCZ2014}.

Nesta seção abordamos alguns detalhes relativos à implementação do
algoritmo de \emph{branch-and-price} para o MWSP baseado na formulação
apresentada na seção anterior. Mencionamos os pré-processamentos
realizados, e as heurísticas utilizadas para cálculo dos limitantes
primais e duais.

\subsection{Pré-processamento}
Os pré-processamentos que realizamos visam fixar, quando possível, o
valor de algumas variáveis associadas às arestas (tanto as que
necessariamente irão fazer parte de qualquer solução como as que
necessariamente não farão parte de nenhuma solução). 

\begin{itemize}
\item \textit{Aresta $t$-essencial}. Lembramos que uma aresta $e$
  de $G$ é $t$-essencial se $G-e$ não é um $t$-spanner de $G$.  Na
  fase de pré-processamento, para cada aresta $t$-essencial $e$ do
  grafo de entrada, impomos que $x_e=1$.

\item \textit{Aresta $t$-inútil}. Dizemos que uma aresta $e=uv \in E$
  é uma \emph{aresta $t$-inútil} de $G$ se \linebreak
  \mbox{$w_e > t \cdot \dist_G(u,v)$}. Para cada aresta $t$-inútil $e$ do
  grafo de entrada, fixamos em zero o valor da variável $x_e$. Observe
  que, no caso unitário, nenhuma aresta é $t$-inútil.
  \end{itemize}

\subsection{Heurísticas para cálculo de limitantes primais}
\label{sec:heuristicas}

Apresentamos nesta subseção duas heurísticas para calcular limitantes
primais, que utilizamos na rotina de B\&B.  Uma delas, a heurística
baseada numa partição em \emph{clusters} só é utilizada no caso
unitário.  Caso uma destas heurísticas encontre um limitante menor do
que o valor da solução incumbente, atualizamos a solução incumbente.
%% limite global.
%Apresentamos a seguir essas heurísticas.

\subsubsection{spanner baseado numa partição em \emph{clusters}}

O Algoritmo~\ref{alg:unweighted_span} descrito a seguir constrói um
spanner de $G$, onde $G$ tem peso unitário,
baseado na partição dos vértices de $G$ em
\emph{clusters}.  Este algoritmo foi desenvolvido por
Peleg~\cite{Peleg2000}.

Dado um inteiro $k \ge 1$, o Algoritmo~\ref{alg:unweighted_span}
constrói um $(2k - 1)$-spanner com no máximo $O(n^{1 + 1/k})$ arestas.
Para o caso unitário e com este fator de dilatação, este algoritmo
constrói um spanner com o menor número de arestas
(veja~\cite{BaswanaKMP2010}).  Ele inicialmente constrói uma partição
dos vértices do grafo (usando o Algoritmo~\ref{alg:basic_part}),
tal que cada classe da partição é um subgrafo conexo, chamado \emph{cluster}.

Cada \emph{cluster} é construído a partir de um vértice e
adicionando-se camadas, finalizando-se a adição de camadas quando
determinada restrição de esparsidade é violada. Após construir a
partição $\partition$ por meio do Algoritmo~\ref{alg:basic_part},
%% o algoritmo \ref{alg:unweighted_span},
para cada \emph{cluster} de $\partition$ é construída uma árvore de
caminhos mínimos. Além disso, para cada \emph{cluster} $S$ de
$\partition$, e para cada vértice $u$ vizinho de $S$ e que não esteja
em $S$, é garantida a existência de uma aresta unindo $u$ a $S$ na
solução final. Tal solução será formada por estas arestas externas ao
\emph{cluster} juntamente com as árvores de caminhos
mínimos. 

%%%%%%%%%%%%%%%%%%

Para $S \subseteq V$, seja $\verneigh(S) = S \cup \{v \in V
\setminus S \;| \;\exists u \in S\text{ e }uv \in E\}$. 


\begin{algorithm}
  \SetAlgoLined
  \textbf{Entrada:} {$G=(V, E)$, onde $|V|=n$, e um inteiro positivo $k$}\;
  \textbf{Saída:} {partição $\partition$ de $V$}\;
  
  $\partition \gets \emptyset$\;
  Seja $G'=(V',E')$ um grafo isomorfo a $G$\;
  \While{$V(G') \neq \emptyset$}{
    Selecione um vértice arbitrário $v' \in V(G')$\;
    $S' \gets \{v'\}$\;
    \While{$|\verneigh(S')| < n^{1/k}|S'|$}{
      $S' \gets \verneigh(S')$\;
    }
    Seja $S = \{v \in V\;|\; \exists\; v' \in S' \text{ e } v' \text{ é o vértice correspondente a } v \text{ no isomorfismo}\}$\;
    $\partition \gets \partition \cup S$\;
    $V(G') \gets V(G') \setminus S'$\;
  }
  \caption{Partição básica} 
  \label{alg:basic_part}
\end{algorithm}

\begin{algorithm}
  \LinesNumbered
  %% \SetAlgoLined
  \textbf{Entrada:} {$G=(V,E)$, inteiro positivo $k$}\;
  \textbf{Saída:} {Um grafo $G'=(V,E')$ que é um $(2k-1)$-spanner de $G$ com no máximo $O(n^{1 + 1/k})$ arestas}\;
  
  \BlankLine
  Construir uma partição $\partition$ de $V$ usando o Algoritmo~\ref{alg:basic_part} (Partição básica)\;

  \ForEach{$S_i \in \partition$}{
     $T_i$ $\gets$ uma árvore de caminhos mínimos de $G[S_i]$
     enraizada em algum vértice de $S_i$;
  }
  $E(G') \gets \bigcup_{S_i \in \partition}E(T_i)$\;
  $\breve{E} \gets \emptyset$\;
  \ForEach{$S_i \in \partition$}{
    \ForEach{$v \in \verneigh(S_i) \setminus S_i$}{
      Seja $u \in S_i$ t.q. $uv \in E$\;
      $\breve{E} \gets \breve{E} \cup uv$\;
    }
  }
  $E(G') \gets E(G') \cup \breve{E}$\;
  %% \For{par de clusters vizinhos $T_i, T_j$}{
  %%   \tcc{a aresta $(T_i,T_j)$ existe no grafo cluster $\tilde{G}(\partition)$}
  %%   Seja $uv \in E$ t.q. $u \in V(T_i)$ e $v \in V(T_j)$\;
  %%   $E' \gets E' \cup \{uv\}$\;
  %% }

  \Return $G'$.
  \caption{spanner baseado em partição em \emph{clusters}} 
  \label{alg:unweighted_span}
\end{algorithm}

\bigskip

O seguinte resultado prova que o Algoritmo~\ref{alg:unweighted_span}
de fato produz a saída que mencionamos (veja o exercício~3 do
  capítulo~16 em Peleg~\cite{Peleg2000}). Parte da prova foi apresentada por 
Peleg~\cite{Peleg2000}, a parte omitida segue abaixo.

\begin{teorema}[Peleg~\cite{Peleg2000}]
  O Algoritmo~\ref{alg:unweighted_span} aplicado a um grafo $G$ de
  ordem $n$, e um inteiro positivo~$k$,  constrói um $(2k-1)$-spanner de
  $G$ com no máximo $O(n^{1+1/k})$ arestas.
  \begin{proof}
    O limite superior relativo ao número de arestas decorre do Teorema~11.5.1~(b)
    em~\cite{Peleg2000} e do fato de que a quantidade total de arestas
    das árvores construídas é menor do que $n^{1 + 1/k}$ (mais especificamente,
    menor ou igual a $n-1$).

    Vamos verificar o fator de dilatação. Seja $G=(V,E)$, e 
    $uv$ uma aresta de $G$. Se $u$ e $v$ pertencem a um mesmo \emph{cluster}, então
    $\dist_{G'}(u,v) \le 2(k-1)$, visto que o raio do \emph{cluster}
    é menor ou igual a $k-1$ (Teorema~11.5.1 (a)
    em~\cite{Peleg2000}). Agora considere \emph{clusters} $S_i, S_j \in \partition$
    tais que $u \in S_i$ e $v \in S_j$.  Se $uv \in \breve{E}$, segue a
    afirmação. Caso contrário, seja $u' \in S_i$ tal que
    $u'v \in E$ (a existência de $u'$ é garantida pela linha 8 do
    Algoritmo~\ref{alg:unweighted_span}) e seja $v_i \in S_i$ o centro de $S_i$. Então
%
%\begin{align*}
$$\dist_{G'}(u,v)\le \dist_{G'}(u,v_i) + \dist_{G'}(v_i,u') + \dist_{G'}(u',v) \le (k-1) + (k-1) + 1 = 2k -1.$$
%\end{align*}    
    \end{proof}
\end{teorema}
Na desigualde anterior, podemos afirmar que $\dist_{G'}(u',v) = 1$ em
decorrência das linhas 7-9 no Algoritmo~\ref{alg:unweighted_span}.

Observamos que, ao executar o Algoritmo~\ref{alg:unweighted_span} ao
longo da construção da árvore de B\&B, nossa implementação prioriza as arestas
correspondentes às variáveis escolhidas para fazer a ramificação e que
foram fixadas com valor um.

\subsubsection{Algoritmo guloso}
No trabalho de Sigurd e Zachariasen~\cite{SigurdZ2004} é mencionado
que os autores estavam interessados em avaliar o desempenho da
heurística gulosa proposta por Alth\"{o}fer et
al.~\cite{AlthoferDDJS1993}, o que os motivou a desenvolver um
algoritmo exato para o MWSP.  Descrevemos a seguir a heurística
gulosa, alvo da comparação, referida como
Algoritmo~\ref{alg:greedy_spanner}.

\medskip

%% \SetInd{0.5em}{0.6em}
%% \SetKwInOut{Input}{Input}
%% \SetKwInOut{Output}{Output}
\begin{algorithm}
  \SetAlgoLined
  \textbf{Entrada:} {$G=(V, E)$ com pesos não negativos nas arestas, número real $t \ge 1$}\;
  \textbf{Saída:} {$t$-spanner $G'$ de $G$ com peso pequeno}\;
  %\BlankLine
  Ordene as arestas de $E$ em ordem não decrescente de seus pesos\;
  $G' \gets (V',E')$, onde $V(G') \gets V, E(G') \gets \emptyset$\;
  \ForEach{$e=uv \in E$}{
    \lIf{$t \cdot w_e < \dist_{G'}(u,v)$}{$E(G')$ $\gets$ $E(G') \cup \{e\}$}
  }
  \Return $G'$.
  \caption{Algoritmo Guloso de Althöfer et al.} 
  \label{alg:greedy_spanner}
\end{algorithm}

A heurística para o cálculo de um limitante primal é baseada no
Algoritmo Guloso de Alth\"{o}fer et al.~\cite{AlthoferDDJS1993}, acima
descrito.  Como o cálculo de limitantes primais é realizado
periodicamente ao longo da execução do algoritmo de B\&B, variáveis
diferentes fixadas em zero ou um implicam (possivelmente) em valores
diferentes de limitantes primais. Em cada nó da árvore, para as arestas
cujas variáveis foram fixadas em um, a heurística as força a pertencer
à solução viável, enquanto que para as arestas cujas variáveis foram
fixadas em zero, a heurística as exclui da solução.  Como a heurística
é executada após a execução do \emph{pricing},
%% Além disso, depois de calcular uma solução ótima através do \emph{pricing}, 
a heurística escolhe as arestas em ordem não decrescente, 
de acordo com os valores das variáveis associadas às arestas.

\subsection{Heurística para cálculo de limitante dual}
Para calcular um limitante dual, implementamos uma heurística que gera
uma árvore geradora mínima (\emph{Minimum Spanning Tree} - MST)
conectando os componentes induzidos pelas arestas cujas variáveis
foram fixadas em um.  Nesta árvore, não consideramos arestas cujas
variáveis foram fixadas em zero. O
Algoritmo~\ref{alg:mst_based_fixed_edges} corresponde à heurística
mencionada. Note que sempre é possível criar uma árvore que conecta os
componentes.  Como veremos na Seção~\ref{sec:nos_filhos}, na criação
dos nós da árvore de B\&B, o nó da árvore que representa a variável
fixada em zero só é criado se não inviabilizar a existência de uma
solução $t$-spanner.
%% implicar na desconecção do grafo.

\begin{algorithm}[t]
  \SetAlgoLined
  \textbf{Entrada:} {$G=(V,E)$, $E_0 \subset E$, $E_1 \subset E$, $w: E \to \espacoRpos$}\;
  \textbf{Saída:} {limitante dual para a tripla $(G,E_0,E_1)$}\;
  % \tcc{$E_0$: Conjunto das arestas fixadas em $0$; $E_1$: conjunto das arestas fixadas em $1$.}
  % \BlankLine
  /* $E_0$: conjunto das arestas fixadas em zero\;
  /* $E_1$: conjunto das arestas fixadas em um\;
   %\BlankLine
  $LD \gets \sum_{e\in E_1}w_e$\;  
  Seja $G_1=(V,E_1)$, e $\comp$ o conjunto dos componentes de $G_1$\;
  Seja $G_{1}^{c} = (V^c_1,E^c_1)$ o grafo assim definido:\\
   $~~ V^c_1 = \{C_i | \; C_i \in \comp\}$\;
   $~~ E^c_1 = \{C_iC_j \; | \; C_i,C_j \in \comp, i \neq j,  \text{ e } \exists x,y \in V(G) \text{ t.q. } x \in V(C_i), y \in V(C_j), xy \in E, xy \notin E_0\}$\;
  %% Seja $E^c_1 = \{xy \in E \setminus (E_0 \cup E_1)\;|\; \exists C_i,C_j \in \comp \text{ t.q. } x \in V(C_i), y \in V(C_j)\}$\;
  $T \gets {\rm MST}(G^c_1)$\; 
  $LD \gets LD + \sum_{e \in E(T)}w_e$\;
  \Return $LD$.   
  \BlankLine 
  %% \eIf{existe $T$}{
  %%   $LD \gets LD + \sum_{e \in E(T)}w_e$\;
  %%   \Return $LD$\;    
  %% }{
  %%   \Return $-1$\;
  %% }
  \caption{MST baseada nas arestas fixadas pelo B\&B} 
  \label{alg:mst_based_fixed_edges}
\end{algorithm}

Caso esta heurística devolva um valor melhor do que o valor gerado pelo
\emph{pricing} (ou seja, o valor gerado pela heurística é maior do que o valor
do \emph{pricing}), então o limitante inferior do nó da árvore sendo analisado
é atualizado com o valor da heurística.
%% \subsection{Escolha da variável fracionária para executar a ramificação}
\subsection{Estratégia de ramificação na árvore de B\&B}
\label{sec:estrategia_ramificacao}
%% Seja $(x^*,Y^*)$ uma solução ótima retornada após a execução do \emph{pricing}.
Adotamos o critério de ramificação em valores inteiros, seguindo a
proposta de Morrison et al.~\cite{MorrisonJSS2016}.  Neste caso, a
ramificação é dividida em duas fases: selecionar uma variável
fracionária onde será feita a ramificação; posteriormente, forçar esta
variável fracionária a ter valores inteiros (no nosso caso, zero ou um)
e, por consequência, criar subproblemas a partir destas restrições de
integralidade. No nosso caso, como as variáveis do programa $\FullPrimalPL$
são binárias e como é escolhida apenas uma variável fracionária, serão
criados (no máximo) dois subproblemas filhos.
%% a ramificação que está sendo feita é 
%% com relação à estratégia de ramificação adotada, 

\subsubsection{Criação dos subproblemas filhos}
\label{sec:nos_filhos}
No processo de ramificação, ao fixar uma variável fracionária em um,
o subproblema resultante é criado e armazenado na lista dos
subproblemas ainda não analisados (veja
Seção~\ref{sec:branch_and_bound}). Para o subproblema resultante da
fixação da variável no valor zero, é feito um teste para saber se o
grafo resultante admite um caminho $t$-spanner entre os extremos da
aresta cuja variável foi fixada em zero. Se o grafo admitir tal
caminho, então o subproblema é criado e armazenado na lista dos
subproblemas ainda não analisados. Caso contrário, o subproblema não é
criado.

\subsubsection{Escolha da variável fracionária}
%% Com relação à estratégia de ramificação,
Implementamos duas formas de escolher uma variável fracionária para
realizar a ramificação:

\begin{itemize}
\item[{\rm (a)}] Escolher uma variável fracionária cujo valor está
  mais próximo de $\frac{1}{2}$.  Como as variáveis são binárias, este
  critério prioriza escolher uma variável para a qual o ``\emph{solver}
  estaria mais indeciso'' sobre o valor a ser atribuído;

\item[{\rm (b)}] Escolher uma variável com um maior valor
  fracionário. Este critério prioriza as variáveis para as quais o
  ``\emph{solver} já está quase certo'' sobre o valor a ser atribuído.

  %% \item Escolhe a variável com menor gap de integralidade, ou seja, $\argmin_{e \in E} (w(e) - (w(e) * x^*(e)))$. Esta forma só serve para os casos;
  \end{itemize}

\subsection{Estratégia de escolha do nó a ser examinado na árvore de B\&B}
\label{sec:befs}
%% ??REFERÊNCIA \cite{MorrisonJSS2016,Clausen1999}

A seleção do próximo nó a ser examinado é feita utilizando a
estratégia de \emph{procurar o melhor primeiro} (\emph{Best-First
  Search} - BeFS)~\cite{MorrisonJSS2016,Clausen1999}. Nesta
estratégia, para cada um dos subproblemas (nós) ativos, devemos
escolher aquele que minimiza uma determinada função. No nosso caso,
esta função corresponde ao limitante inferior gerado pelo
\emph{pricing}. Ou seja, ao longo da execução do algoritmo, vamos
tentando diminuir a diferença entre limitantes primal e dual.
%%  com base nos limites inferiores gerados
%% pelo \emph{pricing}. Nós sempre escolhemos o nós com o menor limite inferior.

\subsection{Processamento do nó da árvore de B\&B}
Seja $v(E_o, E_1)$ o nó que está sendo analisado na árvore de B\&B, onde $E_0$
(resp. $E_1$) corresponde ao conjunto de arestas cujas variáveis em $x$
foram fixadas em zero (resp. um). Seja

  \mbox{$\setOfFixedSpan := \{\incid^{F} \in \espacoE,\; E_0 \cap F =
    \emptyset,\; E_1 \subseteq F\; |\; \text{$G[F]$ é um grafo
      $t$-spanner\}}$}.

\subsubsection{Inclusão de inequações válidas}

Para $e,f \in E,\; e \neq f$, tais que $x_e$ e $x_f$ não foram fixados, seja \\
  $\setOfFixedSpan(e,f) := \{\incid^{F} \in
  \setOfFixedSpan \; |\; \text{$G[F-\{e,f\}]$ é um grafo
    $t$-spanner\}}$.

\noindent Neste caso, a seguinte inequação é válida (e pode ser acrescentada):
%  
\begin{lpformulation}[]
\lpeq[ineq:bridges]{\qquad x_e + x_f \ge 1}{e, f \in E, \; e \neq f,\;
  e,f \notin E_0 \cup E_1,\; \setOfFixedSpan(e,f) = \emptyset}.
\end{lpformulation}

A geração de colunas não será afetada pela inclusão destas inequações,
pois as inequações do programa \FullDualPL\; que poderiam ser violadas
(inequação (\ref{dual:ineq2})) não sofrerão alterações após a inclusão
dessas inequações ao RMP, como pode ser visto a seguir.  Para cada
$e \in E$, seja $F_{ponte}(e)$ o conjunto das arestas relacionadas com
$e$, definida como o conjunto das arestas $f$ tal que o par $e,f$
respeita a restrição~(\ref{ineq:bridges}). Para cada $e \in E$ e
$f \in F_{ponte}(e)$, seja $\beta_{e,f}$ a variável dual associada à
restrição~(\ref{ineq:bridges}).  O dual da relaxação do programa
$\FullPrimalPL$ juntamente com as inequações~(\ref{ineq:bridges}) é
descrito pelo seguinte PL:

\begin{lpformulation}[\FullDualTwoPL]
\lpobj*{max}{\sum_{uv \in E} \sigma_{u,v} + \sum_{e \in E} \sum_{f \in F_{ponte}(e)} \beta_{e,f}}
\lpeq[dual2:ineq1]{\sum_{uv \in E} \pi^{u,v}_{e} + \sum_{f \in F_{ponte}(e)} \beta_{e,f} \le w_e}{e \in E}
\lpeq[dual2:ineq2]{\sum_{e \in E} \delta^{e}_{p}\pi^{u,v}_{e} - \sigma_{u,v} \ge 0}{uv \in E,\, \forall p \in \Pathuv}
\lpeq[]{\pi^{u,v}_{e} \ge 0}{e \in E,\, \forall uv \in E}
\lpeq[]{\sigma_{u,v} \ge 0}{uv \in E}
\lpeq[]{\beta_{e,f} \ge 0}{e \in E,\, \forall f \in F_{ponte}(e)}
\end{lpformulation}


\subsubsection{Tratamento das variáveis fixadas em zero ou um}
Para cada nó da árvore de B\&B, temos um conjunto de arestas que foram
fixadas em zero ou um.  Seja $v(E_0, E_1)$ como descrito
anteriormente. Adicionamos as seguintes equações ao RMP para tratar as
arestas fixadas:

\begin{lpformulation}[]
  \lpeq[ineq:fixed_one]{\sum_{e \in E_1} x_e = |E_1|}{}
  \lpeq[ineq:fixed_zero]{\sum_{e \in E_0} x_e = 0}{}
\end{lpformulation}

Para $e \in E$, vamos considerar o caso em que queremos fixar a
variável $x_e$ em um.  Quando adicionamos a restrição $x_e = 1$, o
\emph{pricing} não é comprometido visto que o grafo de entrada para o
\emph{pricing} não é alterado. Este grafo não necessita ser alterado
pois não é necessário usar a aresta $e$ em nenhum caminho gerado para
ter uma solução válida, visto que a restrição~(\ref{rmp:relate_vars})
para ser válida, não exige que nenhum caminho contenha tal aresta. Uma
alternativa (que não foi implementada) seria não adicionar a restrição
$x_e = 1$. Neste caso, os valores duais gerados (dados pela variável
$\pi$) seriam diferentes e, consequentemente, as arestas teriam custos
diferentes no grafo de entrada do CSPP, mas a estrutura do grafo de
entrada seria a mesma.

  Para $e \in E$, quando queremos fixar a variável $x_e$ em zero, \
  adicionamos a restrição $x_e = 0$ e removemos a aresta $e$ do
  grafo de entrada do CSPP. Uma alternativa (que não foi implementada)
  seria simplesmente remover a aresta $e$ do grafo de entrada do
  CSPP. Neste caso, os valores duais gerados seriam diferentes, mas a
  aresta $e$ continuaria sem poder ser escolhida pelo \emph{pricing}
  para gerar um novo caminho. Em outras palavras, em ambos os casos a
  aresta $e$ nunca poderá ser escolhida como parte de um caminho
  gerado pelo \emph{pricing}.

%% \subsection{Criação dos nós da árvore de BB}
%% \subsection{Estratégia de ramificação na árvore de B\&B}
%% Antes de criar os dois subproblemas a partir de 

\section{Uso do algoritmo para árvore $t$-spanner de peso mínimo}
\label{sec:gc_mwtsp}

Fazendo-se uma pequena alteração, a implementação descrito 
neste capítulo pode ser utilizada para resolver o problema
da árvore $t$-spanner de peso mínimo (MWTSP). Com isso, podemos
comparar essa implementação com as formulações lineares inteiras propostas
por nós para o MWTSP, apresentadas no capítulo anterior. 

Para evitar alterar o \emph{pricing}, não adicionamos ao
programa $\FullPrimalPL$ a restrição que limita o número de arestas da
solução. Em vez disso, adicionamos coeficientes com valores
altos para cada uma das variáveis (associadas às arestas) que compõem
a função objetivo do programa $\FullPrimalPL$.
%% associadas às arestas do grafo.
Os coeficientes escolhidos são suficientemente altos para que qualquer
solução viável com um número menor de arestas seja priorizada com
relação às soluções com um número maior de arestas. Ao fim, se a solução ótima
encontrada tiver um número de arestas maior ou igual ao número de
vértices, então sabemos que o grafo de entrada não admite uma árvore
$t$-spanner.

Toda a implementação (incluindo as heurísticas) para o MWSP continua sendo
válida para o MWTSP. Não encontramos na literatura heurísticas
para cálculo de limitantes primais para o MWTSP.

A implementação de Sigurd e Zachariasen~\cite{SigurdZ2004} não está disponível.
Resolvemos implementar o algoritmo de Sigurd e Zachariasen para realizar
um conjunto maior de experimentos computacionais (quando comparado à quantidade
de
experimentos realizados pelos autores) e para comparar a implementação com as
formulações
lineares inteiras que propusemos no Capítulo~\ref{cap:mwstp} para o MWTSP.
Estas comparações serão apresentadas no Capítulo~\ref{cap:experimentos}.
Como comentado no início do capítulo, Sigurd e Zachariasen não mencionam 
pré-processamentos e heurísticas para cálculo de limitantes primais e duais,
sendo estas as principais diferenças de nossa implementação para a dos
autores.

%% Alguns detalhes de implementação foram implementados para que o algoritmo
%% funcionasse corretamente. 
%% \begin{itemize}
%% \item setSolutionBasedOnRMP(): armazena a solução gerada pelo pricing. Esta solução é inteira. O valor retornado será acrescido de um número grande multiplicado pelo número de arestas da solução;
%% \item compareLB(BranchTreeNode* a, BranchTreeNode* b): utilizada para comparar os unexploredNodes pelo lower bound. Para o caso da árvore, o lower bound retornado será aquele acrescido de um número grande multiplicado pelo número de arestas da solução;
%% \item addVarsXToMasterLP(): a função objetivo atribui um valor alto para cada variável (indiretamente força a solução a ter poucas arestas);
%% \item createBranchNodes: inicia também os real lower bound
%% \item getMSTbasedOnFixedEdges: retorna o custo da MST + (número de arestas multiplicado por uma constante alta);
%% \item basicPart: retorna o custo da solução spanner + (número de arestas multiplicado por uma constante alta);
%% \item greedySpanner: retorna o custo da solução spanner + (número de arestas multiplicado por uma constante alta);
%% \item updateIntegralityGap: para calcular o gap, utiliza os valores reia de globalUB e do lower bound do nó da árvore de busca;
%% \item branchOnX:
%%   \begin{itemize}
%%   \item após receber o valor do price (que está adulterado), calcula o valor real retornado pelo price, sem as coeficientes altos;
%%   \item atualizar o gap de integralidade somente se o lower bound real for menor do que ub real. Isto faz sentido pois o gap de integralidade é calculado pela diferença entre estes valores reais.
%%     \item atualizo o lower bound real com um valor de heurística, caso ele seja melhor do que o calculado pelo pricing
%%     \end{itemize}
%% \item main
%% \end{itemize}


