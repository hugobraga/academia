% !TEX encoding = UTF-8 Unicode
%\chapter{Introdução}% \label{sec:objetivo_geral}
\xchapter{Introdução}{}
\acresetall

	De acordo com \cite{GMCR07}, a capacidade de se adaptar dinamicamente às condições distintas de execução é uma questão muito importante quando se trata de sistemas distribuídos cuja Qualidade de Serviço (\textit{Quality of Service} - QoS) negociada nem sempre pode ser entregue entre os processos. Sistemas distribuídos são definidos como aqueles nos quais os componentes localizados em computadores interligados em rede se comunicam e coordenam suas ações apenas trocando mensagens \cite{COLDOKIN07}. Com relação à degradação da QoS, \cite{JTK00} afirma que esta é quase sempre inevitável e que a reserva de recursos não é suficiente para que a QoS seja sustentada. Levando em consideração este contexto, \cite{GMCR07} propuseram um modelo adaptativo de programação para sistemas distribuídos tolerantes à falhas.
	
	Este modelo (também conhecido como modelo HA - \textit{Hybrid and Adaptative}) é considerado híbrido, visto que é composto por partes assíncronas (nas quais não existem limites temporais para processamento e transferência de mensagens) e síncronas (nas quais existem limites temporais) mantidas por QoS, e adaptativo, visto que a QoS se altera e o modelo necessita se adaptar às condições de QoS. Para que o modelo HA possa funcionar corretamente, ele necessita obter informações sobre a QoS que está sendo provida a cada canal de comunicação.
	
	Tais informações são providas através do dispositivo de software denominado QoS Provider (QoSP). O QoSP corresponde a uma interface padronizada que cria e gerencia canais de comunicação com QoS. O QoSP não é uma arquitetura de QoS, mas sim um invólucro para um conjunto de arquiteturas para prover QoS. Dessa maneira, mesmo que o ambiente de comunicação seja alterado (novas arquiteturas de QoS sejam utilizadas), os processos que usufruem desta interface não sofrerão mudança alguma, visto que a interface do QoSP não muda. O QoSP deve se comunicar com as as arquiteturas de QoS existentes no ambiente para obter informações relativas a QoS de um canal.
	
	A interface do QoSP é definida através de cinco funções: criação de canal, cálculo de \textit{delay}, negociação de QoS, monitoração de QoS e verificação de canal. Apesar das cinco funções, o QoSP pode ser resumido em dois grandes serviços: negociação e monitoração de QoS. O serviço de monitoração engloba as funções de monitoração de QoS e de verificação de canal. A função de monitoração de QoS retorna a QoS que está sendo provida a um canal, enquanto que a função de verificação além de desempenhar tal funcionalidade ela verifica se há tráfego em um canal durante um determinado intervalo de tempo (estas funções serão detalhadas no capítulo \ref{cap:qos_provider}). Uma outra funcionalidade do serviço de monitoração é o monitoramento automático, que corresponde a um monitoramento que é executado mesmo quando não há uma solicitação explícita ao QoSP. 
	
	A monitoração de QoS é um mecanismo utilizado por outros componentes para obter um \textit{feedback} sobre o estado atual da QoS, para que medidas possam ser tomadas baseado neste \textit{feedback}. Isto se aplica bem ao modelo proposto HA, visto que para que o modelo HA possa afirmar que um determinado processo falhou, ele consultar o QoSP (mas especificadamente, o mecanismo de monitoramento). De acordo com \cite{ACH96}, mecanismos de monitoração de QoS são partes fundamentais de arquiteturas de QoS. O módulo (estou utilizando este termo intercambiável com mecanismo) de monitoração do QoSP não é um componente de arquitetura de QoS, mas se comporta como um para o QoSP, visto que é ele quem retorna ao QoS que está sendo provida ao longo dos componentes de software e hardware de um canal \cite{ACH96}. Logo justifica-se a adoção de princípios e de um modelo de sistema de monitoramento.
	
	Este trabalho visa desenvolver, especificar e implementar o mecanismo de monitoramento do QoS Provider, sendo que este último tem como base a arquitetura Diffserv e conta com um sistema operacional de tempo real instalado nos \textit{hosts}. O desenvolvimento deve ser baseado em um modelo e em princípios de sistema de monitoramento, além da utilização de protocolos que diminuam o \textit{overhead} da rede. A utilização de princípios assim como de protocolos que diminuam o \textit{overhead} visa não tornar o sistema de monitoramento um gargalo, mas sim um componente que auxilie a tomada de decisões. O desenvolvimento se dará através da especificação dos algoritmos que compõem o mecanismo de monitoramento assim como de um protocolo utilizado pelo mesmo. A especificação do sistema se dará através da modelagem do mesmo utilizando diagramas UML. Ao final da implementação, o sistema deverá ser testado.