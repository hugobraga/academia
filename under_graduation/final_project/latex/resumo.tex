% !TEX encoding = UTF-8 Unicode
\resumo
%\begin{resumo}

	A capacidade de se adaptar dinamicamente às condições distintas de execução é uma questão muito importante quando se trata de sistemas distribuídos cuja Qualidade de Serviço (\textit{Quality of Service} - QoS) negociada nem sempre pode ser entregue entre os processos \cite{GMCR07}. Dado esta motivação, \cite{GMCR07} propuseram um modelo adaptativo de programação para sistemas distribuídos tolerantes à falhas. Para que este modelo possa funcionar corretamente, ele necessita obter informações sobre a QoS (além da execução de alguns outros serviços) provida aos canais de comunicação. Estas informações são obtidas através da interface padronizada denominada \textit{QoS Provider} (QoSP). Além da padronização, o \textit{QoS Provider} visa encapsular todos os detalhes a respeito das arquiteturas de QoS que estão sendo utilizadas. O QoSP pode ser resumido em dois grandes serviços: negociação e monitoração de QoS. Este trabalho visa desenvolver, especificar e implementar o mecanismo de monitoramento do QoSP. Este mecanismo corresponde ao serviço de monitoramento mencionado anteriormente. Este mecanismo engloba três funcionalidades: monitorar a QoS que está sendo provida a um canal de comunicação, verificar se há tráfego em um canal durante um intervalo de tempo e realizar um monitoramento automático.

\begin{keywords}
Sistemas distribuídos, QoS, monitoramento.
\end{keywords}

%\end{resumo}