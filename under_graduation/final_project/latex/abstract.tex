% !TEX encoding = UTF-8 Unicode
%\begin{abstract}
%	The capability of dynamically adapting to distinct runtime conditions is an important issue when designing distributed systems where negotiated quality of service (QoS) cannot always be delivered between processes \cite{GMCR07}. Given this motivation, \cite{GMCR07} proposed an adaptive programming model for fault-tolerant distributed systems. For this model to function properly it needs to obtain information about the QoS (besides the execution of some other services) provided to the communication channels. This information is obtained through the standardized interface called \textit{QoS Provider} (QoSP). In addition to standardization, the \textit{QoS Provider} aims to encapsulate all the details about the underlying QoS architectures. The QoSP can be summarized in two major services: admission and QoS monitoring. This work aims to develop, implement and specify the QoS monitoring mechanism. This mechanism is the QoS monitoring service mentioned earlier. This mechanism comprises three functions: obtain the current QoS of a channel, assess if message flow happens in a channel within a period of time and carry out a periodic monitoring.
%	
%	\textbf{Keywords}: Distributed systems, Quality of Service (QoS), monitoring.
%\end{abstract}

\begin{otherlanguage*}{english}
\abstract

	The capability of dynamically adapting to distinct runtime conditions is an important issue when designing distributed systems where negotiated quality of service (QoS) cannot always be delivered between processes \cite{GMCR07}. Given this motivation, \cite{GMCR07} proposed an adaptive programming model for fault-tolerant distributed systems. For this model to function properly it needs to obtain information about the QoS (besides the execution of some other services) provided to the communication channels. This information is obtained through the standardized interface called \textit{QoS Provider} (QoSP). In addition to standardization, the \textit{QoS Provider} aims to encapsulate all the details about the underlying QoS architectures. The QoSP can be summarized in two major services: admission and QoS monitoring. This work aims to develop, implement and specify the QoS monitoring mechanism. This mechanism is the QoS monitoring service mentioned earlier. This mechanism comprises three functions: obtain the current QoS of a channel, assess if message flow happens in a channel within a period of time and carry out a periodic monitoring.
		
	% Palavras-chave do resumo em Ingles
\begin{keywords}
Distributed systems, Quality of Service (QoS), monitoring.
\end{keywords}
\end{otherlanguage*}
