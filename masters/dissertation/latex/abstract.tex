\abstract
%\begin{spacing}{1.5}
%\parskip=6pt
Steiner trees are commonly used to model constraints in message multicasting. 
%Typically, a Steiner tree problem involves minimizing the total cost of the tree while satisfying additional constraints, such as bounded maximum node degree.
In this dissertation we address a problem called \emph{Directed k-Spanner with Minimum Degree Steiner Tree Problem} (DSMDStP). This problem
consists in, given a directed weighted graph $G(V,E)$, a \emph{source node} $s \in V$, 
a \emph{stretch factor} $k$ ($k \in \mathbb{R}^+$, $k \ge 1$) and a set of \emph{terminals} $T \subseteq V \setminus \lbrace s \rbrace$, 
finding an arborescence in which the cost (distance) between $s$ and each $t \in T$ 
is less than or equal to $k$ times the shortest distance between $s$ and $t$ in the original graph, 
while minimizing the maximum node out-degree. 
DSMDStP is not approximable sublogarithmically (unless $NP \subset DTIME(n^{\log \log{n}})$).
We describe an approximation algorithm that generates 
an arborescence with out-degree limited to $2\sqrt{|T|} + 2 + O(\log |T|) \cdot d^*$, where $d^*$ is the maximum degree in an optimum solution and 
the arborescence is a spanner with a root-stretch factor of 
$k \cdot \left(1 + \frac{max_{t\in T}\{dist(s,t,G)\}}{min_{t \in T}\{dist(s,t,G)\}}\right)$, 
where $dist(s,t,G)$ represents the shortest distance between $s$ and $t$ in $G$. 
%where $dist_{max} = max\{dist(s,t,G) | t \in T\}$ and $dist_{min} = min\{dist(s,t,G) | t \in T\}$. 
%Although our root-stretch factor is a function of $k$ and extreme terminal distance costs rather than only $k$, in simulations, the spanner guarantee 
%was really good. Actually, in average, the spanner constraint was satisfied. Besides this, the maximum out-degree was low compared to shortest path trees.
Although our root-stretch factor violates $k$, in our experiments the spanner constraint was satisfied or almost satisfied in average. 
Additionally, the resulted out-degree was low.
We also describe a heuristic which provides a root-stretch factor of $k \cdot (\lfloor\sqrt{|T|}\rfloor+2)$ but does
not provide a bound to the out-degree.
%We also describe a heuristic (called SIM) which does not provide a limit to the degree of nodes and has a root-stretch factor of $k \cdot (\sqrt{|T|}+2)$. 
%In our simulations, SIM exhibited interesting characteristics, having a stable behaviour for both degree and spanner guarantee, which contributes to scalability, 
%and also outperforming the other proposed algorithm regarding the degree.
In the experiments, the heuristic was shown to scale well in terms of the maximum degree achieved, and has always outperformed the other algorithms. The heuristic generated additionally a low spanner violation factor.
\begin{keywords}
Steiner tree, Directed Graph, Minimum Degree, Spanner, Approximation, Heuristic
\end{keywords}

%\end{spacing}
