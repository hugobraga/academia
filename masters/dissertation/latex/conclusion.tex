\xchapter{Conclusion}{}
\label{sec:conclusion}
\acresetall

%ALTEREI PARAGRAFO
In this dissertation we presented a version of a Steiner tree problem called \emph{Directed k-Spanner with Minimum Degree Steiner Tree Problem} (DSMDStP). 
Unlike commonly defined Steiner tree problems, in DSMDStP we are interested in minimizing 
the maximum out-degree of the arborescence while respecting the terminals spanner constraints. We assume as input a directed graph.
To the best of our knowledge, these properties characterize \mbox{DSMDStP} as a new problem.
We showed that \mbox{DSMDStP} is not approximable sublogarithmically (unless $NP \subset DTIME(n^{\log \log{n}})$) and described an approximation algorithm and a heuristic to it. 
For both proposed algorithms, we analyzed their complexity.

The approximation algorithm is based on the algorithm described in \cite{Elkin2006}. The algorithm generates an arborescence $\mathcal{A}_f$ 
with maximum degree $2\sqrt{|T|} + 2 + O(\log{|T|})\cdot d^*$, where $d^*$ is the maximum degree of an optimum arborescence.
This seems a reasonable limit, as 
%it was proved in \cite{Fraigniaud2001} that MDST, a restricted version of 
DSMDStP does not admit a sublogarithmic approximation.
Additionally, in $\mathcal{A}_f$ the paths from $s$ to each terminal $t$ have cost $\le k \cdot ( dist(s,t,G) + dist(s,t_{max},G))$, where $t_{max} \in \{t' | t' \in T \land (\forall t'' \in T: dist(s,t'',G) \le dist(s,t',G))\}$. 
Although the algorithm can violate the spanner constraint, 
this only happens to terminals covered in the second phase of the algorithm. 
%In Section \ref{sec:related} we argue that adding the terminal constraints to the problem makes it much harder to approximate.

%ALTEREI PARAGRAFO
The heuristic, called SIM, is based on an iterative application of an algorithm for the MSC problem. SIM does not provide guarantee on maximum node degree. 
However, in our experiments SIM provided lower maximum node out-degree than the approximation algorithm. 
In the final arborescence, the path from $s$ to $t \in T$ has cost $ \le (|T| + 2) \cdot k \cdot dist(s,t,G)$. 
%Different from the approximation algorithm, beyond $k$, the paths' final cost depend only on shortest distance
%, i.e., beyond $k$ it only depends on the cost constraint of $t$. In our simulations, SIM generates trees with maximum degree much lower than the maximum degree in shortest path trees.

The experiments presented quite good results for both proposed algorithms. Regarding the node out-degrees, both the approximation algorithm and SIM yielded low
maximum out-degree and they outperformed the results of a shortest path tree algorithm (SPT). Additionally, SIM has always outperformed the approximation algorithm 
and it has a uniform behaviour, what contributes to scalability. 
On the other hand, the approximation algorithm always outperformed SIM in metrics concerning the spanner constraint. Moreover, in half of the situations 
no violation occurred. The metrics addressing the spanner constraint measured the quality and the quantity of violation. Even when violation occurred, 
these metrics showed that the approximation algorithm violated by low factors. 
Although SIM did not present as good results as the approximation algorithm, 
concerning the metric that addresses the quality of violation (how much the violation occurred), the results were also 
quite good, since on average the violation was by a factor of 1.4 and in worst case by a factor of 2, which are quite acceptable.

%Concerning the spanner property, in average, the approximation algorithm satisfied 
%the spanner constraint, which led us to conclude the theoretical bounds are pessimistic. Even SIM not respecting the spanner constraint, it violated the constraint, 
%in average, by low factors and again exhibited uniform behaviour. So, SIM performed very good for both metrics.

We also described how we can improve (with high probability) the degree guarantee for the approximation algorithm. This can be done through the concepts of \emph{submodular functions} 
and \emph{matroid}. More specifically, we can solve DSMDStP using an instance of a problem called MCG. 
The improved result can be achieved by modelling MCG as a problem of \emph{maximizing a monotone submodular function subject to partition matroid constraint} and using the algorithm presented in \cite{Calinescu2011} to solve it. 

\section{Future Work}
%Although we aimed at proposing a solution for DSMDStP where the spanner constraints were satisfied, we could not do it in this work. 
%So, a possible future work could be to investigate new algorithms to try to improve the resutls. %Our results ...

A possible future work could be to investigate new algorithms to try to improve the resutls, mainly concerning the spanner property. It is important to mention that 
we do not know if DSMDStP admits a solution.

Turning the solutions into distributed ones would be challenging and another possible future work. 
%A distributed solution would allow us to apply it 
%in scenarios where distributed algorithm is a requirement or really important, such as in Wireless Sensor Networks. Besides this, a distributed solution 
Distributed algorithms are necessary in some scenarios, such as in Wireless Sensor Networks. Besides this, a distributed solution 
is more scalable than a centralized one, which is an important property.

We argued in Appendix \ref{sec:matroid} how we could improve the degree guarantee of the proposed approximation algorithm by modelling the problem 
(actually, the subproblem MCG) through the concepts of submodular functions and matroids. But doing this, our algorithm would be a probabilistic one. Although being probabilistic, 
it would be interesting to see by how much the degree is improved in practice. So, another possible future work would be to solve the MCG problem 
through the solution proposed in \cite{Calinescu2011} and compare the results with the ones presented in this dissertation.
%implementar matroid

%\begin{itemize}
  %\item Try to improve the solution in order to respect the spanner constraint.
%\end{itemize}
