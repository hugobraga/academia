%\chapter{An alternative to improve the degree guarantee for the approximation algorithm}
%\chapter{An alternative to solve DSMDStP}
\xchapter{Discussion about how to approach DSMDStP through an alternative way}{}
\label{sec:matroid}

In this appendix we comment on how we could approach the MCG problem through an alternative way and the impact of this new approaching for the DSMDStP. 
This new modeling of MCG would allow us to decrease the maximum degree for the DSMDStP. 
%In this appendix we comment on the impact of solving DSMDStP through an alternative way. 
%In this appendix we discuss how we could improve our degree guarantees for the proposed approximation algorithm. 
In order to do this, we would model 
the MCG problem through the concepts of \emph{submodular functions} and \emph{matroid}. We briefly present these concepts and show how MCG could 
be modeled through them. Although there is a result that allows the improvement of the degree guarantee of our problem, this would turn our algorithm in 
a probabilistic one, as the presented result for this new modeling of MCG is a randomized algorithm. 
The results discussed here are a mere conjecture.

%ALTEREI PARAGRAFO
As mentioned in \cite{Calinescu2011}, the MCG problem \cite{Chekuri2004} could be addressed through the concepts of \emph{monotone submodular functions} and 
\emph{matroids}. More specifically, MCG can be modeled as a problem of maximizing a monotone submodular function under matroid constraint. 
As we can solve MSC using MCG, improvements on solution to MCG extend to MSC as well \cite{Elkin2006} (the SCG problem in \cite{Chekuri2004}). Let $X$ be a ground set of $n$ elements. A function $f: 2^{X} \rightarrow \mathbb{R}^+$ is \emph{submodular} iff: 

\begin{center}
$f(A \cup B) + f(A \cap B) \le f(A) + f(B)$,
\end{center}

for all $A,B \subseteq X$ \cite{Schrijver2003}. $2^X$ represents the set of all subsets of $X$, including the empty set and $X$ itself. 
$f$ is called \emph{monotone} if $f(A) \le f(B)$, for all $A \subseteq B$. In order to address the \emph{matroid} 
concept, besides a ground set $X$ we need the concept of an \emph{independence family} $I \subseteq 2^{X}$, a family of subsets that is downward closed, which 
means that $A \in I$ and $B \subseteq A$ implies that $B \in I$. A \emph{matroid} is a pair $M = (X,I)$ where $I \subseteq 2^{X}$ and

\begin{align*}
&\tag{i} \forall B \in I, A \subset B \Rightarrow A \in I.\\ 
&\tag{ii} \forall A,B \in I; |A| < |B| \Rightarrow \exists x \in B \setminus A; A \cup x \in I.
\end{align*}

The problem addressed in \cite{Calinescu2011} consists in maximizing $f(S)$ over the independent sets $S \in I$. Similar to \cite{Calinescu2011}, 
let us denote this kind of problem by \emph{SUB-M}. The SUB-M problem can be modeled with a matroid constraint 
called \emph{partition matroid}, where $X$ is partitioned 
%\footnote{The MCG does not require the set $X$ to be partitioned into disjoint sets, but in order to solve the MSC problem through the MCG, actually the ground set is partioned into disjoint sets} 
into $l$ subsets $X_1, X_2, ..., X_l$ with associated integers $k_1, k_2, ..., k_l$, 
and $A \subseteq X$ is considered independent iff $|A \cap X_i| \leq k_i$. 

%NOVO PARAGRAFO
Maximizing a monotone submodular function under matroid constraint (SUB-M) was addressed for the first time in \cite{Nemhauser1978,Fisher1978}. 
In these papers, the authors analyze a greedy algorithm and give some approximation results. For SUB-M with partition matroid, the greedy algorithm 
gives a $2$-approximation \cite{Fisher1978}. For the restricted case of SUB-M with uniform matroid ($l = 1$, so the objective is maximizing $f(S): |S| \le k$), the algorithm 
gives a $(e/(e-1))$-approximation. The authors in \cite{Calinescu2011} improved the previous results to an approximation factor of $(e/(e-1))$ for 
any matroid, including the partition matroid, and this is optimal in the oracle model (since this approximation ratio is the best one for the restricted version, 
the one with uniform matroid \cite{Nemhauser1978,Fisher1978}). The authors in \cite{Calinescu2011} improved the previous results through a randomized algorithm. 

%ALTEREI MODIFICADO
The MCG problem can be modeled by SUB-M with partition matroid. Regarding MCG, a solution $H \subset \lbrace S_1, S_2, ..., S_m \rbrace$ 
is independent iff $|H \cap G_i| \leq k_i$ (see problem definition in Section \ref{sec:solve_msc}). Remember that there is a set of partitions $G_1, G_2, ..., G_l$ as input to the MCG problem. For this problem, 
$f: H \subset \lbrace S_1, S_2, ..., S_m \rbrace \rightarrow \mathbb{R}^+$ represents the number of elements of the ground set $X$ covered by 
the elements of $H$. This kind of coverage function is submodular, as mentioned in \cite{Calinescu2011}. 
%This means $f$ is a modular function, which implies it is submodular. 
Based on Theorem 1.1 in \cite{Calinescu2011} and the previous discussion, the following result holds:

\begin{Theo}
  \label{teorema:randomized_algorithm}
  \cite{Calinescu2011}. There is a randomized algorithm which gives an $(e/(e-1))$-approximation
  %\cite{Calinescu2011}. There is a randomized algorithm which gives a $(1 - 1/e)$-approximation
\footnote{Our notation of the approximation factor is different from the one presented in \cite{Calinescu2011}, since the latter is represented by $\frac{1}{f}$, 
where $f$ is the approximation factor in our case.}
(in expectation) to the MCG problem.
\end{Theo}

The expression \emph{in expectation} in theorem \ref{teorema:randomized_algorithm} means the approximation factor is guaranteed with high probability. 
Based on the idea proposed in \cite{Chekuri2004} to solve the MSC problem (the same as the SCG problem in \cite{Chekuri2004}), we can infer that:

\begin{Lem}
  \label{lema:better_approximation_msc}
  %There is an algorithm which gives a $(\log_{\frac{e}{e-1}} n + 1)$-approximation (in expectation) to the MSC problem.
  There is an algorithm which gives a $(\log_{e} n + 1)$-approximation (in expectation) to the MSC problem.
\end{Lem}

So, based on lemma \ref{lema:better_approximation_msc}, theorem \ref{teorema:randomized_algorithm} and theorem \ref{teorema:final_theorem}, we cojecture that %:
%\begin{Conjec}
%  \label{teorema:better_final_result}
\emph{there is an approximation algorithm that generates an arborescence $\mathcal{A}_f$ with (expected) bounded out-degree 
$2\sqrt{k} + 2 + O(\log_{e} l) \cdot d^*$  
and that, with high probability, has paths from $s$ to each terminal $t \in T$ with cost less than or equal to $k \cdot ( dist(s,t,G) + dist(s,t_{max},G))$}.
%\end{Conjec}

As the algorithm proposed in \cite{Calinescu2011} guarantees that the approximation factor holds with high probability, the bounded degree of the aforementioned approximation algorithm is guaranteed in expectation, 
since it depends upon the bound provided by the algorithm to the MCG problem, what is probabilistic in this case, so the expectation follows. Moreover, 
as the authors in \cite{Chekuri2004} calculate the number of iterative runs of the MCG algorithm (and consequently the upper bound to the MSC) for the solution to the MSC problem 
%as the solution to the MSC problem proposed in \cite{Chekuri2004} infer* the number of iterative applications of the MCG algorithm (and consequently infer* the upper bound to the MSC) 
%based on the lower bound (the approximation factor of the algorithm to MCG) of terminals covered each application of the solution to MCG,
based on the approximation factor of the algorithm to MCG,  
by applying the probabilistic solution presented in this chapter, the $(\log_{e} l + 1)$ is an expected upper bound of iterations to cover all terminals.

%As stated in Cojecture \ref{teorema:better_final_result}, the approximation algorithm bounds the out-degree in expectation, 
%since the algorithm proposed in \cite{Calinescu2011} is a probabilistic one rather than a deterministic, which means in this case that the approximation factor holds with 
%high probability. 

%PARAGRAFO NOVO
Although the MCG problem could be modeled as SUB-M with partition matroid and it could be solved by a deterministic algorithm with approximation ratio 
of 2 (the one analyzed in \cite{Nemhauser1978,Fisher1978}), which was the first result, this deterministic algorithm gives the same approximation factor 
of the algorithm presented in \cite{Chekuri2004}.



%\begin{teorema}
%  \label{teorema:randomized_algorithm}
%  \cite{Calinescu2011}. There is a randomized algorithm which gives a $(1 - 1/e)$-approximation (in expectation) to the problem $max \lbrace f(S) : S \in I \rbrace$, 
%where $f: 2^{X} \rightarrow \mathbb{R}^+$ is a monotone submodular function given by a value oracle, and $M = (X,I)$ is a motroid given by a membership oracle.
%\end{teorema}
