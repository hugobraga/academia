\xchapter{Introduction}{}
\acresetall
Given a graph $G=(V,E)$, 
%where $V$ represents a set of nodes and $E$ a set of edges, 
with edge weight function $C : E \to \mathbb{R}^+$,
a source node $s \in V$ and a set of \emph{terminals} $T \subseteq V \setminus \lbrace s \rbrace$, the classical minimum cost Steiner tree problem consists in finding 
a tree $T_r = (V_{T_r},E_{T_r})$ rooted at $s$ that spans $T$ and that has minimum cost (i.e. minimizes $\sum_{e \in E_{T_r}} C(e)$). 
Many variations of this problem have been defined, where additional constraints must be satisfied (e.g., \cite{Oliveira2005,Parsa1998,Wang2009}).

In the context of communication networks, the Steiner Tree problem is frequently related to multicasting, where a message has to be sent to
a subset of the nodes. %network nodes.
% Typically minimizing the cost of the tree results in minimizing bandwidth consumption \ref{??}. 
Constraints are added to the problem, for example, to address quality-of-service requirements.
In real-time applications, maximum transmission delays for a message sent from $s$ to each node in $T$ must have to be satisfied \cite{Oliveira2005,Nguyen2008}. 
Maximum delays can be modeled by defining maximum costs for paths from $s$ to each node in $T$. 
Alternatively, restrictions in communication delays can be defined as a limit on the increase of the distance of a pair of nodes on the tree when compared 
to the distance of the same pair of nodes in the original graph, i.e., by imposing a \emph{stretch factor}. This is commonly modeled as a 
spanner property.
%Generally, the same maximum cost is defined for all nodes in $T$. However, scenarios where different maximum costs
%are defined for different nodes also exist \cite{Parsa1998,Zhengying2001}.

An additional common constraint is a bounded node degree. Limiting node degree is of practical importance, for example,  
to address the number of connections that can be managed by communication devices such as routers and switches or by communication protocols 
(as in Bluetooth), to minimize the effort of replicating data in multicast operations \cite{Oliveira2005,Nguyen2008}, or to
decrease topology maintenance in mobile networks. In the context of wireless sensor networks, minimizing node degree is particularly important, due to 
the need for keeping resource use at
a minimum (as sensor nodes are typically scarce-resource devices).

In order to address degree minimization and a spanner property between the source node and terminals, we define in this dissertation a new problem called
\emph{Directed k-Spanner with Minimum Degree Steiner Tree Problem} (DSMDStP). Instead of minimizing 
the total cost of the tree, we are interested in minimizing the maximum out-degree of nodes. 
%Trees with limited maximum degree are called \emph{narrow} trees. 
Additionally, we aim at generating a tree such that, given a stretch factor $k$, the distance between the source node and a terminal on this tree
is at most $k$ times the minimum distance between them in the original graph. 
%Additionally, we aim to limit the stretch factor in the paths between the source node and the terminals, i.e., the ratio of the distance between the source node $s$ and a terminal node $t$ in the final tree 
%and the distance between the same nodes in the original graph is limited. 
%Trees with limited stretch factor are called \emph{shallow} trees. 
%previous related work, as, for example, \cite{Fraigniaud2001})}
We call Steiner trees with this kind of parameterized spanner property \emph{single-sink k-spanner Steiner trees} 
(although we are not minimizing the tree cost, the nomenclature \emph{Steiner problem} is consistent, for example, with \cite{Fraigniaud2001}). 

DSMDStP is not approximable within $(1-\varepsilon)\log_e{n}$, for any $0 < \varepsilon < 1$ (unless $NP \subset DTIME(n^{\log \log{n}})$).
We describe first an approximation algorithm for DSMDStP based on the algorithm presented in \cite{Elkin2006}. Our algorithm provides a guarantee on the 
%maximum out-degree of nodes and gives a stretch factor of $k \cdot (1 + \frac{dist_{max}}{dist_{min}})$, 
maximum out-degree of nodes and gives a stretch factor of $k \cdot \left(1 + \frac{max_{t\in T}\{dist(s,t,G)\}}{min_{t \in T}\{dist(s,t,G)\}}\right)$, 
where %: 
%$dist_{max} = max_{t \in T}\{dist(s,t,G)\}$; $dist_{min} = min_{t \in T}\{dist(s,t,G)\}$; and 
$dist(u,v,G)$ represents the cost of a minimum cost directed path from $u$ to $v$ in $G$.
Although our algorithm does not guarantee the desired stretch factor of $k$, 
in our experiments the cost of the generated paths is much lower than the provided upper bound. In fact, on average,
%(measured by a metric defined by us)
the spanner constraint was satisfied on half of the scenarios and almost satisfied on the other half. 
%Additionally, the average maximum degree can be decreased as long as the given stretch factor $k$ is increased??. %This happens due to the way the algorithm works*.

We also describe a heuristic for the problem, called \emph{Sliced and Iterative Multiple Set-Cover} (SIM). The heuristic does not provide guarantee on node degree and 
provides a root-stretch factor of $k \cdot (\lfloor\sqrt{|T|}\rfloor+2)$. In the experiments, SIM exhibited quite good results. Its curve is uniform for 
maximum out-degree, which is an important requirement for scalability. Moreover, it always outperformed the approximation algorithm 
in relation to maximum node degree. Regarding the violation of spanner constraint, SIM was outperformed by the approximation algorithm but the violation 
occurred by a low factor (in average by 1.4 and by 2 in the worst case).

%Although a shortest path tree (SPT) would be a trivial approach to satisfy the spanner constraint, 
%our proposed algorithms always outperformed SPT regarding the degree and the approximation algorithm mimicked SPT's behaviour regarding the spanner property, 
%which is really good since SPT always generates spanner with stretch factor of 1.

%Although we are not minimizing the cost of the tree, the nomenclature \emph{Steiner problem}
%is consistent with, for example, \cite{Fraigniaud2001}.

This dissertation is organized as follows.
In Chapter \ref{sec:problem} we formally define \mbox{DSMDStP} and show that it is not approximable sublogarithmically (unless $NP \subset DTIME(n^{\log \log{n}})$).
In Chapter \ref{sec:algorithms} we describe the proposed algorithms for DSMDStP. 
In Chapter \ref{sec:evaluation} we describe the results of experiments performed to evaluate the algorithms. 
In Chapter \ref{sec:related} we discuss related work. %, including the violation of terminal constraints in the algorithms. 
Finally, Chapter \ref{sec:conclusion} concludes the dissertation. 
%In Appendix \ref{sec:matroid} we present a way to improve the degree guarantee (in expectation) through the concepts of submodular function and matroids.
In Appendix \ref{sec:matroid} we comment how to approach DSMDStP through an alternative way, 
which comprises the concepts of submodular function and matroids.    

%In the next chapter, we are going to discuss extensively the related work, presenting the main results. Moreover, we are going to argue why DSMDStP is 
%a new problem, giving the differences between our problem and the related ones.
In the next chapter we define DSMDStP as well as to prove that it is not approximable sublogarithmically.
